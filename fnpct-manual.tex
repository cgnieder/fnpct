% arara: pdflatex: { shell: on , interaction: nonstopmode }
% arara: biber
% arara: pdflatex
% arara: pdflatex
% --------------------------------------------------------------------------
% the FNPCT package
% 
%   footnote kerning
% 
% --------------------------------------------------------------------------
% Clemens Niederberger
% Web:    https://github.com/cgnieder/fnpct/
% E-Mail: contact@mychemistry.eu
% --------------------------------------------------------------------------
% Copyright 2012--2022 Clemens Niederberger
% 
% This work may be distributed and/or modified under the
% conditions of the LaTeX Project Public License, either version 1.3c
% of this license or (at your option) any later version.
% The latest version of this license is in
%   http://www.latex-project.org/lppl.txt
% and version 1.3c or later is part of all distributions of LaTeX
% version 2008/05/04 or later.
% 
% This work has the LPPL maintenance status `maintained'.
% 
% The Current Maintainer of this work is Clemens Niederberger.
% --------------------------------------------------------------------------
\documentclass{fnpct-manual}

\setfnpct{
  add-trailing-token = [0pt]:[-.05em]{colon} ,
  after-comma-space = -0.1em ,
  after-dot-space =  -0.1em ,
  before-comma-space = -0.05em ,
  before-dot-space =  -0.05em ,
  before-footnote-space = 0.05em
}

\addbibresource{cnltx.bib}
\addbibresource{\jobname.bib}
\addbibresource{biblatex-examples.bib}
\begin{filecontents*}{\jobname.bib}
@book{bringhurst04,
  title     = {The Elements of Typographic Style},
  author    = {Robert Bringhurst},
  year      = {2004},
  version   = {3.2},
  isbn      = {978-0-88179-205-5},
  publisher = {Hartley \&\ Marks, Canada}
}
\end{filecontents*}

\AdaptNote\footcite{m}{#NOTE{#1}}

\newname\bringhurst{Robert Bringhurst}
\newname\lefloch{Bruno Le Floch}

\setlength\parfillskip{0pt plus .8\textwidth}

\begin{document}

\section{Preface for those upgrading from version 0.x}

\begin{bewareofthedog}
  For those upgrading from versions~0 to version~1: a number of options have
  been dropped or renamed. Please have a close look at your log file for
  warnings.  Also, the way commands are adapted and hence the syntax of
  \cs{AdaptNote} has changed completely. In many if not all cases leaving them
  as they were will lead to errors. \par
  The \cs*{mult...} variants are not defined by default anymore. They can be
  retained through the load-time option \option{multiple} if needed.  Letting
  the note commands behave as their \cs*{mult...} variants is not supported
  any more. \par
  \cs{innernote} is not provided anymore. Depending on user feedback I might
  try to add it again to \fnpct. \par
  Please read this manual carefully so you detect all changes. \par
  \emph{I am sorry for the inconvenience but I am convinced that the new
    syntax is more powerful and more flexible in the long run.}
\end{bewareofthedog}

\section{License}
\license

\section{What's it all about?}
\subsection{Introduction}
The purpose of this package is to offer kerning for footnote marks, \ie\ the
superscripts. This is not appropriate for all superscripts.  Symbols must be
handled differently than numbers.  And of course the amount depends on the
chosen font.  \bringhurst\ says in \citetitle{bringhurst04}:

\begin{cnltxquote}[\emph{\citetitle{bringhurst04}} \cite{bringhurst04}]
  Superscripts frequently come at the ends of phrases or sentences.  If they
  are high above the line, they can be kerned over a comma or period, but this
  may endanger readability, especially if the text is set in a modest size.
\end{cnltxquote}

\fnpct\ can not make these decisions for you.  It sets some initial values for
the comma and the full stop which looked good to me with the tested fonts.
Additionally it kerns the superscripts away from words when it follows
directly.  The amounts of the kerning can be changed using options.

\fnpct\ also switches the order of the superscript and the following comma or
full stop.  Additional punctuation marks can be added to the switching
behavior and the amount of kerning can be set for each punctuation mark
individually.  This behavior can also be turned off.

\subsection{Basics}
The \fnpct\ package basically does two things to footnotes:
\begin{itemize}
  \item if footnote marks are followed by a comma or a full stop\footnote{More
    punctuation marks can be added through a package option.} the order of
  footnote and punctuation mark is reversed
\item and the kerning gets adjusted.
\end{itemize}
As a side effect a new method of creating multiple
footnotes\multfootnote{like;this} is provided.

In what way is the kerning adjusted?  After being placed behind the
punctuation mark the footnote mark is moved a little bit back, namely by the
amount specified with the option \option{after-punct-space}.  If the footnote
mark follows a word \emph{without} being followed by a
punctuation\footnote{Well, it does not necessarily have to follow a
  \emph{word}.  The important point is \emph{not being followed} by a
  punctuation mark.}, there (obviously) is no order switching and a little
space is inserted before the footnote mark, namely the amount specified by the
option \option{before-footnote-space}.

Now, let's see some action:
\begin{example}
  The three little pigs built their houses out of straw\footnote{not to be
    confused with hay}, sticks\footnote{or lumber according to some sources}
  and bricks\footnote{probably fired clay bricks}.
\end{example}

To ensure that the kerning is set the right way the footnote \emph{must} be
placed \emph{before} the full stop or the comma. \emph{The command can look
  ahead but not look back}.  This means if you place the \cs{footnote} command
after a full stop or a comma it is treated as if following a word, \ie\ a thin
space is inserted: effectively the opposite of the desired behavior.  The
effects are demonstrated in figure~\vref{fig:effects}.

The order-switching can be prevented using the option \option{reverse} since
not all countries and languages have the same typographic conventions.  In
this case the full stop and the comma are moved a bit back, namely by the
amount specified with the option \option{before-punct-space}.

\begin{figure}
  \centering
  \begin{tabular}{>{\setfnpct{dont-mess-around}}ll}
    \toprule
      without \fnpct &
      with \fnpct \\
    \midrule
      \strut\quad text.\footnotemark[1] &
      \strut\quad text\footnotemark[1]. \\
      \strut\quad text\footnotemark[1]. &
      \setfnpct{reverse}\strut\quad text\footnotemark[1]. \\
      \strut\quad text\footnotemark[1] &
      \strut\quad text\footnotemark[1] \\
    \bottomrule
  \end{tabular}
  \caption{The effect of \fnpct.}\label{fig:effects}
\end{figure}

\subsection{Temporarily disable (or enable) switching -- new syntax of the
  \cs*{footnote} command}
One maybe want to put some footnote marks \emph{before} the punctuation and
some after, for example because the first one describes a single word but the
second one a whole sentence.  For this purpose \fnpct\ adds a \sarg\ argument
to \cs{footnote} and \cs{footnotemark}.  The complete new syntax now is as
follows:
\begin{commands}
  \command{footnote}[\sarg\oarg{num>}\marg{footnote text}\meta{tpunct}]
    new \sarg\ argument added.  \meta{tpunct} is the optional trailing
    punctuation mark.
  \command{footnotemark}[\sarg\oarg{num>}\meta{tpunct}]
    new \sarg\ argument added.  \meta{tpunct} is the optional trailing
    punctuation mark.
\end{commands}
The \sarg\ argument temporarily turns off the punctuation/footnote switching.
In case you set \keyis{reverse}{true} the \sarg\ argument temporarily
\emph{enables} the switching.

\begin{example}
  \begin{minipage}{.4\linewidth}
    \noindent The three little pigs built their houses
    out of straw\footnote*{not to be confused with hay},
    sticks\footnote{or lumber according to some sources}
    and bricks\footnote{probably fired clay bricks}.
  \end{minipage}\hfil
  \setfnpct{reverse}
  \begin{minipage}{.4\linewidth}
    \noindent The three little pigs built their houses
    out of straw\footnote*{not to be confused with hay},
    sticks\footnote{or lumber according to some sources}
    and bricks\footnote{probably fired clay bricks}.
  \end{minipage}
\end{example}

\section{Options}\label{sec:options}
\begin{commands}
  \command{setfnpct}[\marg{options}]
    All available options are listed below. They are set with \cs{setfnpct}.
    \default{Underlined} values are set if the option is used without value.
\end{commands}
\begin{options}
  \keybool{dont-mess-around}\Default{false}
    Switches of \fnpct's mechanism.
  \keybool{reverse}\Default{false}
    This option reverses the behavior of starred and un-starred footnotes.
  \keybool{ranges}\Default{false}
    Activates ranges for multiple footnotes.  This is not compatible with
    \pkg{hyperref} and does not work for the marks of some packages.
  \keylit{add-trailing-token}{\oarg{before}\meta{token}\oarg{after}\marg{name}}
    Adds \meta{token} to \fnpct's mechanism and activates it. \meta{before}
    sets the default value for the newly created option
    \option{before-\meta{name}-space}. \meta{after} sets the default value for
    the newly created option \option{after-\meta{name}-space}.
    Instead of the option you can also use the command \cs*{AddPunctuation},
    see section~\ref{sec:adapt}.
  \keyval{activate-trailing-tokens}{tokenlist}\Default{.,}
    Activates tokens that have been added with \option{add-trailing-token}.
  \keyval{deactivate-trailing-tokens}{tokenlist}
    Deactivates tokens that have been added with \option{add-trailing-token}.
  \keyval{before-footnote-space}{dimension}\Default{.06em}
    Sets the space that is inserted before a footnote that \emph{does not
      follow a punctuation or other note} to \meta{dimension}.
  \keyval{before-dot-space}{dimension}\Default{-.16em}
    Sets the space that is inserted before a dot if the dot follows a footnote
    mark to \meta{dimension}.
  \keyval{before-comma-space}{dimension}\Default{-.16em}
    Sets the space that is inserted before a comma if the dot follows a footnote
    mark to \meta{dimension}.
  \keyval{before-punct-space}{dimension}
    Sets the space that is inserted before any punctuation mark known to
    \fnpct\ if the punctuation mark follows a footnote mark to
    \meta{dimension}.
  \keyval{after-dot-space}{dimension}\Default{-.06em}
    Sets the space that is inserted after a dot if the dot precedes a footnote
    mark to \meta{dimension}.
  \keyval{after-comma-space}{dimension}\Default{-.06em}
    Sets the space that is inserted after a comma if the dot precedes a
    footnote mark to \meta{dimension}.
  \keyval{after-punct-space}{dimension}
    Sets the space that is inserted after any punctuation mark known to
    \fnpct\ if the punctuation mark precedes a footnote mark to
    \meta{dimension}.
  \keyval{mult-fn-delim}{code}\Default{\code{;}}
    Sets the symbol that divides the seperate notes in the \cs*{mult...}
    variants.
  \keyval{separation-symbol}{value}\default{\code{,}}
    Redefines \cs{multfootsep} to \meta{value}.  This symbol separates
    multiple footnotes if \keyis{ranges}{false}.
  \keyval{range-symbol}{value}\Default{\code{--}}
    Redefines \cs{multfootrange} to \meta{value}.  This symbol separates
    multiple footnotes if \keyis{ranges}{true}.
  \keyval{print-separation}{code}\Default{\cs*{textsuperscript}\Marg{\cs*{multfootsep}}}
    The actual code placed between multiple footnotes if \keyis{ranges}{false}.
  \keyval{print-range}{code}\Default{\cs*{textsuperscript}\Marg{\cs*{multfootrange}}}
    The actual code placed between multiple footnotes if \keyis{ranges}{true}.
\end{options}

% \subsection{Language Specific Settings}
% Package \pkg{babel}'s~\cite{pkg:babel} French settings redefine
% \cs*{@footnotemark}.  \fnpct\ resets this redefinition but sets the
% \option{before-footnote-space} equal to a thin space (\code{.16667em}).

% In case the automatic kerning doesn't work for whatever reason the spaces are
% available as user commands as well:
% \begin{commands}
%   \command{kfp}
%     Insert \option{after-dot-space} or (if \keyis{punct-after}{true})
%     \option{before-dot-space}. 
%   \command{kfc}
%     Insert \option{after-comma-space} or (if \keyis{punct-after}{true})
%     \option{before-comma-space}.
% \end{commands}

\section{Multiple footnotes}\label{sec:multiple}
\subsection{Basics}
Since multiple footnotes have to be treated differently\footnote{see these
  footnotes}\footnote{for an example}, \fnpct\ recognizes if \cs*{footnote} is
used multiple times in a row:

\begin{example}
  The three little pigs built their houses out of straw\footnote{not to be
    confused with hay}, sticks\footnote{or lumber according to some sources}
  and bricks\footnote{probably fired clay bricks}\footnote{or something else}.
\end{example}

\subsection{Ranges}
Starting with a minimum number of three footnotes in a row \fnpct\ can also
display the footnote marks as a range.
\begin{example}
  \setfnpct{ ranges = true }
  The three little pigs built their houses out of straw\footnote{not to be
    confused with hay}, sticks\footnote{or lumber according to some sources}
  and bricks\footnote{probably fired clay bricks}%
  \footnote{or something else}\footnote{what do I know}.
\end{example}
It is important to note that this has limits: ranges do not work well with
\pkg{hyperref}.  For that reason \fnpct' disables ranges if it detects that
\pkg{hyperref} has been loaded.  If you insist on using ranges together with
\pkg{hyperref} use
\begin{options}
  \keybool{keep-ranges}\Default{false}
    to enable them again.
\end{options}

\subsection{The delimiter and separator}
Some packages or classes like the \KOMAScript\ classes define
\cs{multfootsep}.  If it is not predefined it is defined by \fnpct:
\begin{commands}
  \command{multfootsep}\Default{,}
    In its default definition this command simply expands to a comma.
  \command{multfootrange}\Default{--}
    In its default definition this command simply expands to an endash.
\end{commands}
This commands can be redefined to adapt to your needs. You can do this with
\cs*{renewcommand} or by using these options:
\begin{options}
  \keyval{separation-symbol}{value}\default{\code{,}}
    Redefines \cs{multfootsep} to \meta{value}.  This symbol separates
    multiple footnotes if \keyis{ranges}{false}.
  \keyval{range-symbol}{value}\Default{\code{--}}
    Redefines \cs{multfootrange} to \meta{value}.  This symbol separates
    multiple footnotes if \keyis{ranges}{true}.
\end{options}
Both symbols are placed inside \cs*{textsuperscript}. This can also be
redefined if necessary:
\begin{options}
  \keyval{print-separation}{code}\Default{\cs*{textsuperscript}\Marg{\cs*{multfootsep}}}
    The actual code placed between multiple footnotes if \keyis{ranges}{false}.
  \keyval{print-range}{code}\Default{\cs*{textsuperscript}\Marg{\cs*{multfootrange}}}
    The actual code placed between multiple footnotes if \keyis{ranges}{true}.
\end{options}

\subsection{\cs*{mult...} variants}

\begin{bewareofthedog}
  The commands presented in this section are defined to keep some backwards
  compatibility to versions of \fnpct\ prior to v1.0.  In order to use them
  you have to call the package with
  \cs*{usepackage}\Oarg{multiple}\Marg{fnpct}.
\end{bewareofthedog}
In earlier versions of \fnpct\ detecting following footnotes was not as
reliable as it is now.  For that reason it defined the following command:
\begin{commands}
  \command{multfootnote}[\sarg\Marg{\meta{list};\meta{of};\meta{footnotes}}\meta{tpunct}]
    Different footnotes are separated with a semicolon.  The \sarg\ turns the
    footnote/punctuation switching temporarily off.
\end{commands}
\begin{example}
  The three little pigs built their houses out of straw\footnote{not to be
    confused with hay}, sticks\footnote{or lumber according to some sources}
  and bricks\multfootnote{probably fired clay bricks;or something else}.
\end{example}
This command splits its content at every occurrence of \code{;} and creates a
footnote for each.  This delimiter can be changed:
\begin{options}
  \keyval{mult-fn-delim}{code}\Default{;}
    Sets the symbol that divides the seperate notes in the \cs*{mult...}
    variants.
\end{options}

Since the semicolon might be part of the footnote text you might have some
trouble.  Enclosing it in braces should work:
\begin{example}
  The three little pigs built their houses out of straw\footnote{not to be
    confused with hay}, sticks\footnote{or lumber according to some sources}
  and bricks\multfootnote{probably fired clay bricks{;} or something else;what
    do I know}.
\end{example}

\section{Extend \fnpct}
\subsection{Additional punctuation marks}
So if you want to extend the punctuation switching and kerning to other punctuation
marks you can do something like this:
\begin{sourcecode}
  \setfnpct{ add-trailing-token = !{bang} }
\end{sourcecode}
\setfnpct{ add-trailing-token = !{bang} }
After that the bang is recognized and footnotes switch position:
\begin{example}
  The three little pigs built their houses out of straw\footnote{not to be
    confused with hay}? Sticks\footnote{or lumber according to some sources}
  and bricks\footnote{probably fired clay bricks}!
\end{example}

New options have also been defined:
\begin{example}
  \setfnpct{ after-bang-space = .03em }
  The three little pigs built their houses out of straw\footnote{not to be
    confused with hay}? Sticks\footnote{or lumber according to some sources}
  and bricks\footnote{probably fired clay bricks}!
\end{example}

And option \option{after-punct-space} is obeyed:
\begin{example}
  \setfnpct{ after-punct-space = 2pt }
  The three little pigs built their houses out of straw\footnote{not to be
    confused with hay}? Sticks\footnote{or lumber according to some sources}
  and bricks\footnote{probably fired clay bricks}!
\end{example}

\begin{bewareofthedog}
  Please note that you should not use \option{add-trailing-token} inside a
  group. Ideally it is used in the document preamble only.  Using it inside of
  a group can lead to unwanted side effects and should not be attempted.
\end{bewareofthedog}

\setfnpct{ deactivate-trailing-tokens = ! }

\subsection{Make note commands known to \fnpct}\label{sec:adapt}
If a package is not natively supported you can try and adapt commands
yourself.  Before you do so please check section~\vref{sec:other:packages}.
\begin{commands}
  % \AdaptNote
  \command{AdaptNote}[\Marg{\cs*{cs}}\marg{arguments}\oarg{counter name}\marg{code}] 
    This redefines \cs*{cs}.  How the redefinition works is best demonstrated
    with an example. See below for an example.  The arguments \meta{arguments}
    are defined with the same syntax as \pkg{xparse}'s
    \cs*{NewDocumentCommand}.  See the \pkg{xparse} manual~\cite{pkg:xparse}
    for a complete description.  Within \meta{code} \code{\#NOTE} is replaced
    with the original command.  If the optional argument is not used the
    counter name \code{cs} for a note command \cs*{cs} is assumed.

    In addition to the arguments specified in \meta{arguments} an optional
    star as first argument is defined which reverses \fnpct's behavior.
    \meta{code} is also placed inside a group.
  % \AdaptNoteName
  \command{AdaptNoteName}[\marg{csname}\marg{arguments}\oarg{counter name}\marg{code}]
    Like \cs{AdaptNote} but the first argument expects a csname instead of a
    command sequence token.
  % \MultVariant
  \command{MultVariant}[\Marg{\cs*{cs}}]
    Defines a command \cs*{multcs} which splits is argument at each occurence
    of the token set by the option \option{mult-fn-delim} and turns it into a
    series of calls of \cs*{cs}.  \cs*{multcs} also has a starred variant in
    order to create a series of starred calls of \cs*{cs}.
  % \MultVariantName
  \command{MultVariantName}[\marg{csname}]
    Like \cs{MultVariant} but the argument expects a csname instead of a
    command sequence token.
  % \AddPunctuation
  \command*{AddPunctuation}[\oarg{before}\meta{token}\oarg{after}\marg{name}]
    Adds \meta{token} to \fnpct's mechanism and activates it. \meta{before}
    sets the default value for the newly created option
    \option{before-\meta{name}-space}. \meta{after} sets the default value for
    the newly created option \option{after-\meta{name}-space}.  This is the
    same as using option \option{add-trailing-token}.
\end{commands}

\fnpct\ adapts \LaTeX's standard footnote commands the following
way\footnote{Using \cs*{@mpfn} as counter name is essential to make ranges
  work in minipages.}:
\begin{sourcecode}
  \AdaptNote\footnote{o+m}[\@mpfn]{\IfNoValueTF{#1}{#NOTE{#2}}{#NOTE[#1]{#2}}}
  \AdaptNote\footnotemark{o}{\IfNoValueTF{#1}{#NOTE}{#NOTE[#1]}}
  \MultVariant\footnote
  % KOMA-Script's or memoir's \footref:
  \AdaptNote\footref{m}{#NOTE{#1}}
\end{sourcecode}
You essentially have to rebuild the syntax of the original command and place
\code{\#NOTE} in the code where it should be called.  This may seem cumbersome
at first but allows to adapt commands with a more or less arbitrary kind of
syntax.

The command \cs{AdaptNote} can be used multiple times on the same command in
order to change a previous redefinition.

\section{Other packages}\label{sec:other:packages}
\fnpct\ tries its best to support other footnote and related
packages\footnote{If you find some package missing please let me know.}.  Each
of the following subsections is dedicated to one of these packages and if and
how they work together with \fnpct.  Fortunately most of them do quite well.

Most notably the \option{ranges} option cannot deal with some packages, often
due to the lack of a suiting \cs*{...text} 

\subsection{bigfoot}\label{sec:bigfoot}
The package \pkg{bigfoot}~\cite{pkg:bigfoot} is supported.

You need to be a bit cautious, though: \emph{verbatim material won't work
  inside footnotes anymore}.  You might use \lefloch's \pkg{cprotect}
package~\cite{pkg:cprotect} if you really need verbatim material in footnotes
\emph{and} want to use \fnpct.  Since \fnpct\ does not redefine any
\cs{footnotetext} like command it will still work inside one of them.

\subsection{endnotes}\label{sec:endnotes}
The package \pkg{endnotes}~\cite{pkg:endnotes} is supported.

\subsection{enotez}\label{sec:enotez}
The package \pkg{enotez}~\cite{pkg:enotez} is supported.

\subsection{fixfoot}\label{sec:fixfoot}
The \pkg{fixfoot}~\cite{pkg:fixfoot} package provides a possibility to create
repeating footnotes.  \fnpct\ supports this package provided you take care of
the following:

\emph{Use \cs{DeclareFixedFootnote} only in the preamble but \emph{after}
  loading \fnpct.}

\cs*{DeclareFixedFootnote}\Marg{\cs*{cs}}\marg{footnote text} is used to store
the \meta{footnote text} in \cs*{cs} which in turn creates a footnote mark for
it but doesn't repeat the footnote text on the same page in the bottom.  The
document needs several runs to get all the numbers and footnotes right.

Every fixed footnote declared with \cs{DeclareFixedFootnote} gets an optional
\sarg\ to prevent the punctuation switching.

\noranges{fixfoot}

\subsection{footmisc}\label{sec:footmisc}
The \pkg{footmisc} package~\cite{pkg:footmisc} provides a range of options to
customize footnotes, for example output them as margin notes or count
footnotes per page.

Testing showed no incompatibilities with \pkg{footmisc}.  The only thing is
that you won't have to (and shouldn't) use its \option{multiple} option.

\subsection{footnote}\label{sec:footnote}

Testing showed no incompatibilities with \pkg{footnote}.  It cannot adapt its
environment \env*{footnote}, though.

\subsection{manyfoot}\label{sec:manyfoot}
The package \pkg{manyfoot}~\cite{pkg:manyfoot} is supported.

\subsection{pagenote}\label{sec:pagenote}
The package \pkg{pagenote}~\cite{pkg:pagenote} is supported.

\noranges{pagenote}

\subsection{parnotes}\label{sec:parnotes}
The package \pkg{parnotes}~\cite{pkg:parnotes} is supported.

\noranges{parnotes}

\subsection{sepfootnotes}\label{sec:sepfootnotes}
The package \pkg{sepfootnotes}~\cite{pkg:sepfootnotes} is supported.  Define
new footnote types \emph{after} \fnpct\ has been loaded.

\subsection{sidenotes}\label{sec:sidenotes}
\begin{bewareofthedog}
  The package \pkg{sidenotes}~\cite{pkg:sidenotes} is \emph{not} supported!
\end{bewareofthedog}
\fnpct\ is not able to adapt \pkg{sidenotes}' commands due to the current
implementation of \pkg{sidenotes}.

\subsection{snotez}\label{sec:snotez}
The package \pkg{snotes}~\cite{pkg:snotez} is supported.

\subsection{tablefootnote}\label{sec:tablefootnote}
The package \pkg{tablefootnote}~\cite{pkg:tablefootnote} is supported.

\noranges{tablefootnote}

\subsection{tufte-latex's \cs*{sidenote}}\label{sec:tufte}
The \cs*{sidenote} command from the \pkg*{tufte-latex} classes
\cls{tufte-book} and \cls{tufte-handout} is supported.

\noranges*{tufte-latex}

More precisely: the \cs*{sidenote} command from \pkg*{tufte-latex} does not
work with \option{ranges}. But since \cs*{sidenote} is only a synonym for
\cs*{footnote} for \pkg*{tufte-latex} you can use the latter.

\subsection{biblatex's \cs*{footcite}}\label{sec:biblatex}
If you're willing to sacrifice \cs*{footcite}'s \sarg\ argument (which it has
in some styles) you can adapt the commands provided by
\pkg{biblatex}~\cite{pkg:biblatex} as well:
\begin{sourcecode}
  \AdaptNote\footcite{oo+m}[footnote]{%
    \setfnpct{dont-mess-around}%
    \IfNoValueTF{#1}
      {#NOTE{#3}}
      {\IfNoValueTF{#2}{#NOTE[#1]{#3}}{#NOTE[#1][#2]{#3}}}%
  }
\end{sourcecode}
You could use another token for the star, though, if needed:
\begin{sourcecode}
  \AdaptNote\footcite{t+oo+m}[footnote]{%
    \setfnpct{dont-mess-around}%
    \IfBooleanTF{#1}{%
      \IfNoValueTF{#2}
        {#NOTE{#4}}
        {\IfNoValueTF{#3}{#NOTE[#2]{#4}}{#NOTE[#2][#3]{#4}}}%
    }{%
      \IfNoValueTF{#2}
        {#NOTE*{#4}}
        {\IfNoValueTF{#3}{#NOTE*[#2]{#4}}{#NOTE*[#2][#3]{#4}}}%
  }
\end{sourcecode}
Testing showed no problems whatsoever but \fnpct\ does not adapt the command
itself.

\begin{bewareofthedog}
  Turning \fnpct\ off inside of \cs*{footcite} is advised as shown in above
  examples.  \cs*{footcite} calls \cs*{footnote} internally which else might
  lead to inconsistent and unwanted spaces.
\end{bewareofthedog}


\appendix

\printbibliography

\end{document}
TODO:
- memoir, \mutfootsep
- \thanks
- French
- nested notes and \AdaptText\footnotetext{om}{\IfNoValueTF{#1}{#NOTE{#2}}{#NOTE[#1]{#2}}}
