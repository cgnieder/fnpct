% --------------------------------------------------------------------------
% the FNPCT package
% 
%   footnote kerning
% 
% 2012/05/20
% --------------------------------------------------------------------------
% Clemens Niederberger
% Web:    https://bitbucket.org/cgnieder/fnpct/
% E-Mail: contact@mychemistry.eu
% --------------------------------------------------------------------------
% Copyright 2012 Clemens Niederberger
% 
% This work may be distributed and/or modified under the
% conditions of the LaTeX Project Public License, either version 1.3
% of this license or (at your option) any later version.
% The latest version of this license is in
%   http://www.latex-project.org/lppl.txt
% and version 1.3 or later is part of all distributions of LaTeX
% version 2005/12/01 or later.
% 
% This work has the LPPL maintenance status `maintained'.
% 
% The Current Maintainer of this work is Clemens Niederberger.
% --------------------------------------------------------------------------
% The fnpct package consists of the files
%  - fnpct.sty,
%  - README
% --------------------------------------------------------------------------
% If you have any ideas, questions, suggestions or bugs to report, please
% feel free to contact me.
% --------------------------------------------------------------------------
% the package is inspired by the following question on TeX.SE:
%   http://tex.stackexchange.com/q/56094/5049
% --------------------------------------------------------------------------
% if you want to compile this documentation you'll need the document class
% `cnpkgdoc' which you can get here:
%    https://bitbucket.org/cgnieder/cnpkgdoc/
%    the class is licensed LPPL 1.3 or later
\documentclass[toc=index,toc=bib]{cnpkgdoc}
\docsetup{
  pkg      = fnpct ,
  code-box = {
    skipbelow = .5\baselineskip ,
    skipabove = .5\baselineskip ,
  }
}
\addcmds{
  arabic,
  DeclareNewFootnote,
  endnote,
  footnoteB,
  kfc,
  kfp,
  multendnote,
  multfootnote,
  multparnote,
  myfn,
  parnote,
  parnotes,
  setfnpct,
  theendnotes,
  thempfootnote
}
\setfnpct{multiple}
\usepackage{endnotes}
\usepackage{parnotes}

\renewcommand\thempfootnote{\arabic{mpfootnote}}

\usepackage[backend=biber]{biblatex}
\addbibresource{biblatex-examples.bib}
\addbibresource{\jobname.bib}

% rudimentary solution for a `maintainer' field:
\DeclareFieldFormat{authortype}{\mkbibparens{#1}}
\renewbibmacro*{author}{%
  \ifboolexpr{
    test \ifuseauthor
    and
    not test {\ifnameundef{author}}
  }
    {\printnames{author}%
     \iffieldundef{authortype}
       {}
       {\setunit{\space}%
	\usebibmacro{authorstrg}}}
    {}}

\begin{filecontents}{\jobname.bib}
@software{bigfoot,
  title   = {bigfoot},
  author  = {David Kastrup},
  date    = {2006/07/15},
  version = {1.25},
  url     = {http://www.ctan.org/pkg/bigfoot}
}
@software{endnotes,
  title   = {endnotes},
  author  = {Robin Fairbairns},
  authortype = {current maintainer},
  date    = {2012/01/15},
  url     = {http://www.ctan.org/pkg/endnotes}
}
@software{fixfoot,
  title   = {fixfoot},
  author  = {Robin Fairbairns},
  date    = {2007/12/12},
  version = {0.3a} ,
  url     = {http://www.ctan.org/pkg/fixfoot}
}
@software{footmisc,
  title   = {footmisc},
  author  = {Robin Fairbairns},
  date    = {2011/06/06},
  version = {5.5b},
  url     = {http://www.ctan.org/pkg/footmisc}
}
@software{footnote,
  title   = {footnote},
  author  = {Mark Wooding},
  date    = {1997/01/28},
  version = {1.13},
  url     = {http://www.ctan.org/pkg/footnote}
}
@software{manyfoot,
  title   = {manyfoot},
  author  = {Alexander I. Rozhenko},
  date    = {2005/09/11},
  version = {1.10},
  url     = {http://www.ctan.org/pkg/manyfoot}
}
@software{parnotes,
  title   = {parnotes},
  author  = {Micheal Hughes},
  date    = {2012/03/01},
  version = {1},
  url     = {http://www.ctan.org/pkg/parnotes}
}
@software{pagenote,
  title   = {pagenote},
  author  = {Will Robertson},
  authortype = {current maintainer},
  date    = {2009/09/03},
  version = {1.1a},
  url     = {http://www.ctan.org/pkg/pagenote}
}
@software{tablefootnote,
  title   = {tablefootnote},
  author  = {H.-Martin M\"{u}nch},
  date    = {2012/01/14},
  version = {1.0g},
  url     = {http://www.ctan.org/pkg/tablefootnote}
}
@software{yafoot,
  title   = {yafoot},
  author  = {Hiroshi Nakashima},
  date    = {1999/07/14},
  version = {1.0},
  url     = {http://www.ctan.org/pkg/yafoot}
}
\end{filecontents}

\begin{filecontents}{\jobname.ist}
 preamble "\\begin{theindex}\n Section titles are indicated \\textbf{bold},
 packages \\textsf{sans serif}, commands \\code{\\textbackslash\\textcolor{code}{brown}}
 and options \\textcolor{key}{\\code{green}}.\\newline\n\n"
 heading_prefix "{\\bfseries "
 heading_suffix "\\hfil}\\nopagebreak\n"
 headings_flag  1
 delim_0 "\\dotfill "
 delim_1 "\\dotfill "
 delim_2 "\\dotfill "
 delim_r "\\textendash"
 suffix_2p "\\nohyperpage{\\,f.}"
 suffix_3p "\\nohyperpage{\\,ff.}"
\end{filecontents}

\makeindex
\begin{document}

\section{License and Requirements}
\fnpct is placed under the terms of the LaTeX Project Public License,
version 1.3 or later (\url{http://www.latex-project.org/lppl.txt}).
It has the status \enquote{maintained}.

\fnpct depends on the \paket{l3kernel} and \paket*{xparse} which is part of the
\paket{l3packages} bundle.

\newpage
\section{What's it all about?}
\subsection{Basics}\secidx{basics}
The \fnpct package basically does two things to footnotes: if footnote marks are
followed by a comma or a dot the order of footnote and punctuation mark is reversed
and the kerning gets adjusted.

All examples in this documentation use
\begin{beispiel}[code only]
 \renewcommand\thempfootnote{\arabic{mpfootnote}}
\end{beispiel}

Now, let's see some action:
\begin{beispiel}
 \begin{minipage}{.4\linewidth}
  \noindent The three little pigs built their houses
  out of straw\footnote{not to be confused with hay},
  sticks\footnote{or lumber according to some sources}
  and bricks\footnote{probably fired clay bricks}.
 \end{minipage}
\end{beispiel}

To ensure that the kerning is set the right way the footnote \emph{must} be placed
\emph{before} the dot or the comma. The command can look ahead but not look back.
This means if you place the \cmd{footnote} command after a dot or a comma it is
treated as if following a word, \ie a thin space is inserted: effectivly the opposite 
of the desired behaviour.

The order-switching can be prevented using a package option since not all countries
and languages have the same typographic conventions.
\begin{beschreibung}
 \option{punct-after}{\default{true}/false} when \code{true} the punctuation sign
 will be placed \emph{after} the footnote.
\end{beschreibung}
which like all options can also be set using the setup command.
\begin{beschreibung}
 \befehl{setfnpct}{<options>} set up options. Can be used anywhere in the document.
 Some options can only be set in the preamble, though.
\end{beschreibung}
\begin{beispiel}
 \setfnpct{punct-after}
 \begin{minipage}{.4\linewidth}
  \noindent The three little pigs built their houses
  out of straw\footnote{not to be confused with hay},
  sticks\footnote{or lumber according to some sources}
  and bricks\footnote{probably fired clay bricks}.
 \end{minipage}
\end{beispiel}
\secidx*{basics}

\subsection{Temporarily disable switching}\secidx{disable switching}
One maybe want to put some footnote marks \emph{before} the punctuation and some
after, for example because the first one describes a single word but the second
one a whole sentence. For this purpose \fnpct adds a \code{*} argument to \cmd{footnote}
and \cmd{footnotemark}:
\begin{beschreibung}
 \befehl[footnotea]{footnote}*[<num>]{<footnote text>} new \code{*} argument added.
 \befehl[footnotemarka]{footnotemark}*[<num>] new \code{*} argument added.
\end{beschreibung}
This argument temporarily turns off the punctuation/footnote switching.
\begin{beispiel}
 \begin{minipage}{.4\linewidth}
  \noindent The three little pigs built their houses
  out of straw\footnote*{not to be confused with hay},
  sticks\footnote{or lumber according to some sources}
  and bricks\footnote{probably fired clay bricks}.
 \end{minipage}
\end{beispiel}

\achtung{Please be aware that this package is in an experimental state and there
hasn't been extensive testing with other footnote behaviour changing packages!}
\secidx*{disable switching}

\section{Options}\secidx{options}
All package options are listed below. They all can be set as a package option or
with the \cmd{setfnpct} command. Most of them are for adjusting the kerning.
\begin{beschreibung}
 \option{after-dot-space}{<dim>} space to be inserted after a dot and before the
 footnote mark. Default = \code{-.06em}
 \option{after-comma-space}{<dim>} space to be inserted after a comma and before
 the footnote mark. Default = \code{-.06em}
 \option{after-punct-space}{<dim>} set both spaces.
 \option{punct-after}{\default{true}/false} when \code{true} the punctuation sign
 will be placed \emph{after} the footnote.
 \option{before-dot-space}{<dim>} space to be inserted before a dot and after the
 footnote mark, \ie with \key{punct-after}{true}. Default = \code{-.15em}
 \option{before-comma-space}{<dim>} space to be inserted before a comma and after
 the footnote mark, \ie with \key{punct-after}{true}. Default = \code{-.15em}
 \option{before-punct-space}{<dim>} set both spaces.
 \option{before-footnote-space}{<dim>} space to be inserted between a word and the
 following footnote mark. Default = \code{.06em}
 \option{dont-mess-around} sets all mentioned lengths to \code{0} and \key{punct-after}{true}.
 Basically this is as if you hadn't loaded \fnpct except the multiple footnote
 command is still available, see section \ref{sec:multiple}.
 \option{multiple}{\default{true}/false} lets \cmd{footnote} be equal to
 \cmd{multfootnote}
 \option{mult-fn-delim}{<delimiter>} sets the delimiter for the \cmd{multfootnote}
 command.
 \option{mult-fn-sep}{<separator>} sets the separator between multiple footnote
 marks.
 \option{normal-marks}{\default{true}/false} sets the footnote marks in the foot
 not as superscripts but as normal font. If you're not using a \paket[koma-script]{KOMA-Script}
 class this option will load the package \paket*{scrextend}. \emph{This option can
 only be set in the preamble}. Default = \code{false}
\end{beschreibung}

Let's take a look at an example with some ridiculous settings:
\begin{beispiel}
 % some ridiculous settings:
 \setfnpct{after-punct-space=2pt,before-footnote-space=2pt}
 \begin{minipage}{.4\linewidth}
  \noindent The three little pigs built their houses
  out of straw\footnote{not to be confused with hay},
  sticks\footnote{or lumber according to some sources}
  and bricks\footnote{probably fired clay bricks}.
 \end{minipage}
\end{beispiel}

And now the same with switched order:
\begin{beispiel}
 % some ridiculous settings:
 \setfnpct{punct-after,before-punct-space=2pt,before-footnote-space=2pt}
 \begin{minipage}{.4\linewidth}
  \noindent The three little pigs built their houses
  out of straw\footnote{not to be confused with hay},
  sticks\footnote{or lumber according to some sources}
  and bricks\footnote{probably fired clay bricks}.
 \end{minipage}
\end{beispiel}

Some of the options are explained in a bit more detail in the next sections.
\secidx*{options}

\section{Multiple footnotes}\secidx{multiple footnotes}\label{sec:multiple}
\subsection{Basics}\secidx[basics]{multiple footnotes}
Since multiple footnotes have to be treated differently\footnote{see these footnotes;for an example},
\fnpct provides an extra command for that:
\begin{beschreibung}
 \befehl{multfootnote}*{<list>;<of>;<footnotes>} different footnotes are separated
 with a semicolon. The \code{*} turns the footnote/punctuation switching temporarily
 off.
\end{beschreibung}
\begin{beispiel}
 \begin{minipage}{.4\linewidth}
  \noindent The three little pigs built their houses
  out of straw\footnote{not to be confused with hay},
  sticks\footnote{or lumber according to some sources}
  and bricks\multfootnote{probably fired clay bricks;or
  something else}.
 \end{minipage}
\end{beispiel}

Every of the items of the list has an optional argument equivalent to the optional
argument of \cmd{footnote}:
\begin{beispiel}
 \begin{minipage}{.4\linewidth}
  \noindent The three little pigs built their houses
  out of straw\footnote{not to be confused with hay},
  sticks\footnote{or lumber according to some sources}
  and bricks\multfootnote{probably fired clay bricks;[5]or
  something else}.
 \end{minipage}
\end{beispiel}

Additionally every item has an optional \code{*} which only invokes \cmd{footnotetext}.
This enables for example to set nested footnotes\secidx[nested footnotes]{multiple footnotes}
without disrupting the multiple setting. The following example is shown in figure
\ref{fig:nested}:
\begin{beispiel}[code only]
\documentclass{article}
\usepackage[
  paperwidth=.5\textwidth,
  paperheight=12\baselineskip,
  margin=5pt,
  bottom=1.5cm]{geometry}

\usepackage[multiple]{fnpct}

\begin{document}
  \noindent The three little pigs built their houses
  out of straw\footnote{not to be confused with hay},
  sticks\footnote{or lumber according to some sources}
  and bricks\multfootnote{probably fired clay bricks%
  \footnotemark;*or something else;what do I know}.
\end{document}
\end{beispiel}

\begin{figure}[ht]
 \centering
 \includegraphics{nested_ex.pdf}
 \caption{nested footnotes}\label{fig:nested}
\end{figure}


\subsection{The delimiter and separator}\secidx[delimiter]{multiple footnotes}\secidx[separator]{multiple footnotes}
Since the semicolon might be part of the footnote text you might have some trouble.
But there are ways around. Maybe try enclosing it with braces:
\begin{beispiel}
 \begin{minipage}{.4\linewidth}
  \noindent The three little pigs built their houses
  out of straw\footnote{not to be confused with hay},
  sticks\footnote{or lumber according to some sources}
  and bricks\multfootnote{probably fired clay bricks{;}
  or something else;what do I know}.
 \end{minipage}
\end{beispiel}

There are options which lets you choose the (input) delimiter and the (output)
separator:
\begin{beschreibung}
 \option{mult-fn-delim}{<delimiter>} choose delimiter for the \cmd{multfootnote}.
 Default = \code{;}
 \option{mult-fn-sep}{<separator>} choose the separator that is put between footnote
 marks. Default = \code{,}
\end{beschreibung}
\begin{beispiel}
 \setfnpct{mult-fn-delim=//,mult-fn-sep=;}
 \begin{minipage}[t]{.4\linewidth}
  \noindent The three little pigs built their houses
  out of straw\footnote{not to be confused with hay},
  sticks\footnote{or lumber according to some sources}
  and bricks\multfootnote{probably fired clay bricks;
  or something else//what do I know}.
 \end{minipage}\hfil
 \setfnpct{mult-fn-delim=;,mult-fn-sep=}
 \renewcommand*\thempfootnote{\fnsymbol{mpfootnote}}
 \begin{minipage}[t]{.4\linewidth}
  \noindent The three little pigs built their houses
  out of straw\footnote{not to be confused with hay},
  sticks\footnote{or lumber according to some sources}
  and bricks\multfootnote{probably fired clay bricks;
  or something else}.
 \end{minipage}
\end{beispiel}

\subsection{Automagic}\secidx[automagic]{multiple footnotes}
If you want you can turn all footnotes into \cmd{multfootnote}s.
\begin{beschreibung}
 \option{multiple}{\default{true}/false} let \cmd{footnote} behave like \cmd{multfootnote}.
 \emph{This option can only be set in the preamble}. Default = \code{false}
\end{beschreibung}
\begin{beispiel}
 % in preamble: \setfnpct{multiple}
 % or \usepackage[multiple]{fnpct}
 \begin{minipage}{.4\linewidth}
  \noindent The three little pigs built their houses
  out of straw\footnote{not to be confused with hay},
  sticks\footnote{or lumber according to some sources}
  and bricks\footnote{probably fired clay bricks;or
  something else}.
 \end{minipage}
\end{beispiel}
\secidx*{multiple footnotes}

\section{Footcites}\secidx{footcites}
Preliminary testing suggests that \fnpct seems to be compatible with the
\cmd{footcite} and \cmd{footfullcite} commands provided by the \paket{biblatex}
package. However, the punctuation switching does not apply to them.

The kerning works \ldots
\begin{beispiel}
 \setfnpct{before-footnote-space=2pt}
 \begin{minipage}{.5\linewidth}
  \noindent The three little pigs built their houses
  out of straw\footnote{not to be confused with hay},
  sticks\footnote{or lumber according to some sources}
  and bricks\footnote{probably fired clay bricks}. The
  companion\footfullcite{companion} has nothing on this
  topic.
 \end{minipage}
\end{beispiel}

\ldots\ but the punctuation switching doesn't:
\begin{beispiel}
 \begin{minipage}{.5\linewidth}
  \noindent The three little pigs built their houses
  out of straw\footnote{not to be confused with hay},
  sticks\footnote{or lumber according to some sources}
  and bricks\footnote{probably fired clay bricks}. The
  companion has nothing on this topic\footfullcite{companion}.
 \end{minipage}
\end{beispiel}

One can simulate that behaviour by actively setting one of these commands:
\begin{beschreibung}
 \befehl{kfp} insert the \key{after-dot-space} or (if \key{punct-after}{true})
 the \key{before-dot-space}.
 \befehl{kfc} insert the \key{after-comma-space} or (if \key{punct-after}{true})
 the \key{before-comma-space}.
\end{beschreibung}
\begin{beispiel}
 \setfnpct{after-punct-space=-.15em}
 \begin{minipage}{.5\linewidth}
  \noindent The three little pigs built their houses
  out of straw\footnote{not to be confused with hay},
  sticks\footnote{or lumber according to some sources}
  and bricks\footnote{probably fired clay bricks}. The
  companion has nothing on this topic.\kfp\footfullcite{companion}
 \end{minipage}
\end{beispiel}

In case of \key{punct-after}{true}:
\begin{beispiel}
 \setfnpct{punct-after}
 \begin{minipage}{.5\linewidth}
  \noindent The three little pigs built their houses
  out of straw\footnote{not to be confused with hay},
  sticks\footnote{or lumber according to some sources}
  and bricks\footnote{probably fired clay bricks}. The
  companion has nothing on this topic\footfullcite{companion}\kfp.
 \end{minipage}
\end{beispiel}
\secidx*{footcites}

\section{Other packages}\secidx{other packages}
\fnpct tries it best to support other footnote and related packages. Each of the
following subsections is dedicated to one of these packages and how they work
together with \fnpct.

\subsection{bigfoot}\label{ssec:bigfoot}\secidx[bigfoot]{other packages}
The \paket{bigfoot} \cite{bigfoot} package extends the possibilities of the
\paket{manyfoot} package \cite{manyfoot}, see section \ref{ssec:manyfoot}. The main
feature is to use different classes of footnotes which are typeset in different
\enquote{apparatus} (or layers as I like to call them) on the bottom of the page.

\fnpct is compatible with \paket{bigfoot}. You need to be a bit cautious, though.
You need to
\begin{itemize}
 \item load \paket{bigfoot} first;
 \item declare footnotes with \cmd{DeclareNewFootnote} \emph{after} loading \fnpct
 but \emph{in} the document preamble.
\end{itemize}
Also verbatim material won't work inside footnotes anymore. Since \fnpct does not
redefine any \cmd{footnotetext} like command it will still work inside one of them.

For every footnote class defined with \cmd{DeclareNewFootnote} the commands
\cmd{footnote<class>} and \cmd{footnotemark<class>} are redefined with the starred
variant and a \cmd{multfootnote<class>} is defined. The \key{multiple} option
will turn all \cmd{footnote<class>} commands into the corresponding
\cmd{multfootnote<class>}.

\fnpct has an additional package option which \emph{cannot} be set with \cmd{setfnpct}
and only has any effects if \paket{bigfoot} has been loaded:
\begin{beschreibung}
 \option{bigfoot-default-top}{\default{true}/false} sets the \code{default} footnote
 class as top layer. Default = \code{false}
\end{beschreibung}

Since this package cannot easily combine every footnote package for demonstration
purposes the following code is shown in figure \ref{fig:bigfoot}:
\begin{beispiel}[code only]
\documentclass{article}
\usepackage[
  paperwidth=.5\textwidth,
  paperheight=12\baselineskip,
  margin=5pt,
  bottom=1.5cm]{geometry}

\usepackage{bigfoot}
\usepackage[bigfoot-default-top]{fnpct}
\setfnpct{multiple}
\DeclareNewFootnote[para]{B}[alph]

\begin{document}

\noindent The three little pigs built their houses
out of straw\footnote*{not to be confused with hay%
\footnoteB{let alone grass}}, sticks\footnote{or
lumber according to some sources} and bricks%
\footnote{probably fired clay bricks;or something}.

\end{document}
\end{beispiel}

\begin{figure}[ht]
 \centering
 \includegraphics{bigfoot_ex.pdf}
 \caption{\paket*{bigfoot} example}
 \label{fig:bigfoot}
\end{figure}

\subsection{endnotes}\secidx[endnotes]{other packages}
The \paket{endnotes} \cite{endnotes} package povides the commands \cmd{endnote}
and \cmd{endnotemark} which can be used to output all notes at the end of a chapter,
say.

If the package is loaded \fnpct extends its functionality to these commands in
the same way it does with footnotes. For example an according \cmd{multendnote}
is defined. With the package option \key{multiple} all \cmd{endnote}s are turned
into \cmd{multendnote}s.
\begin{beispiel}
 \begin{minipage}[t]{.4\linewidth}
  \noindent The three little pigs built their houses
  out of straw\endnote{not to be confused with hay},
  sticks\endnote{or lumber according to some sources}
  and bricks\endnote{probably fired clay bricks}.
  
  \theendnotes
 \end{minipage}\hfil
 \begin{minipage}[t]{.4\linewidth}
  \noindent The three little pigs built their houses
  out of straw\endnote*{not to be confused with hay},
  sticks\endnote{or lumber according to some sources}
  and bricks\multendnote{probably fired clay bricks;or
  something else}.
  
  \theendnotes
 \end{minipage}
\end{beispiel}

\subsection{fixfoot}\secidx[fixfoot]{other packages}
The \paket{fixfoot} \cite{fixfoot} package provides possibility to create repeating
footnotes. \fnpct supports this package provided you take care of the following:
\begin{itemize}
 \item load the \paket{fixfoot} package first;
 \item use \cmd{DeclareFixedFootnote} only in the preamble but \emph{after} loading
 \fnpct;
\end{itemize}
\cmd{DeclareFixedFootnote}{\cmd{cs}}\ma{<footnote text>} is used to store the
\ma{<footnote text>} in \cmd{cs} and create a footnote mark for it but don't
repeat the footnote text on the same page in the bottom. The document needs
several runs to get all the numbers and footnotes right.

Every fixed footnote declared with \cmd{DeclareFixedFootnote} gets an optional
\code{*} to prevent the punctuation switching.

Since this package cannot easily combine every footnote package for demonstration
purposes the following code is shown in figure \ref{fig:fixfoot}:
\begin{beispiel}[code only]
\documentclass{article}
\usepackage[
  paperwidth=.55\textwidth,
  paperheight=12\baselineskip,
  margin=5pt,
  bottom=1.5cm]{geometry}
  
\usepackage{fixfoot}
\usepackage{fnpct}

\DeclareFixedFootnote{\myfn}{I'm confused: what do I want to say?}

\begin{document}

\noindent The three little pigs\myfn\ built their houses
out of straw\myfn*, sticks\footnote{or lumber according
to some sources} and bricks\footnote{probably fired clay
bricks}.

\end{document}
\end{beispiel}

\begin{figure}[ht]
 \centering
 \includegraphics{fixfoot_ex.pdf}
 \caption{\paket*{fixfoot} example}
 \label{fig:fixfoot}
\end{figure}

\subsection{footmisc}\label{ssec:manyfoot}\secidx[footmisc]{other packages}
The \paket{footmisc} package \cite{footmisc} provides a range of options to customize
footnotes, for example output them as margin notes or count footnotes per page.

Testing showed no incompatibilities with \paket{footmisc}. The only thing is that
you won't have to use its \code{multiple} option.

\subsection{footnote}\secidx[footnote]{other packages}
Unfortunately \fnpct is not compatible with the \paket{footnote} package \cite{footnote}.
Or more precisely it is not compatible if the command pair \cmd{savenotes}/\cmd{spewnotes}
is invoked. This disables \paket{footnote}'s environments as well as its environment
escaping mechanism through \cmd{makesavenoteenv}.

\subsection{manyfoot}\secidx[manyfoot]{other packages}
The \paket{manyfoot} package \cite{manyfoot} is not and will not directly be
supported. It is loaded by \paket{bigfoot} (see section \ref{ssec:bigfoot}), anyway.
And since the latter states this in its documentation
\begin{zitat}[David Kastrup \cite{bigfoot}]
 Purpose of this package is to provide a one-stop solution to almost all problems
 related to footnotes. You can use it as a drop-in replacement of the `\paket{manyfoot}'
 package, but without many of its shortcomings, and quite a few features of its
 own.
\end{zitat}
you should probably prefer it anyway.

\subsection{pagenote}\secidx[pagenote]{other packages}
The package \paket{pagenote} \cite{pagenote} is supported and its \cmd{pagenote}
command is treated the same way as all other note commands: it got an optional
\code{*} argument and an additional \cmd{multpagenote} is defined.

\subsection{parnotes}\secidx[parnotes]{other packages}
The \paket{parnotes} package \cite{parnotes} does something similar to the \paket{endnotes}
package. basically it allows to output the footnote text after a paragraph, either
by using a special environment or by invoking \cmd{parnotes}.

If \paket{parnotes} is loaded \fnpct extends its functionality analogous
to the one of \paket{endnotes}.

Again the \key{multiple} option turns all \cmd{parnote}s into \cmd{multparnote}s.
\begin{beispiel}
 \begin{minipage}[t]{.4\linewidth}
  \noindent The three little pigs built their houses
  out of straw\parnote{not to be confused with hay},
  sticks\parnote{or lumber according to some sources}
  and bricks\parnote{probably fired clay bricks}.
  
  \parnotes
 \end{minipage}\hfil
 \begin{minipage}[t]{.4\linewidth}
  \noindent The three little pigs built their houses
  out of straw\parnote*{not to be confused with hay},
  sticks\parnote{or lumber according to some sources}
  and bricks\multparnote{probably fired clay bricks;or
  something else}.
  
  \parnotes
 \end{minipage}
\end{beispiel}

\subsection{tablefootnote}\secidx[tablefootnote]{other packages}
The package \paket{tablefootnote} \cite{tablefootnote} is supported and its
\cmd{tablefootnote} command is treated the same way as all other note commands:
it got an optional \code{*} argument and an additional \cmd{multtablefootnote}
is defined.

\subsection{yafoot}\secidx[yafoot bundle]{other packages}
Currently no issues are known when using \fnpct with one or all of the packages
of the \paket{yafoot} \cite{yafoot} bundle. Just for reference: these packages are
\paket{pfnote}, \paket{fnpos} and \paket{dblfnote}.
\secidx*{other packages}

\appendix
\printbibliography
\printindex
\end{document}