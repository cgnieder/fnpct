% arara: pdflatex: { shell: on , interaction: nonstopmode }
% !arara: biber
% !arara: pdflatex
% !arara: pdflatex
% --------------------------------------------------------------------------
% the FNPCT package
% 
%   footnote kerning
% 
% --------------------------------------------------------------------------
% Clemens Niederberger
% Web:    https://github.com/cgnieder/fnpct/
% E-Mail: contact@mychemistry.eu
% --------------------------------------------------------------------------
% Copyright 2012--2020 Clemens Niederberger
% 
% This work may be distributed and/or modified under the
% conditions of the LaTeX Project Public License, either version 1.3c
% of this license or (at your option) any later version.
% The latest version of this license is in
%   http://www.latex-project.org/lppl.txt
% and version 1.3c or later is part of all distributions of LaTeX
% version 2008/05/04 or later.
% 
% This work has the LPPL maintenance status `maintained'.
% 
% The Current Maintainer of this work is Clemens Niederberger.
% --------------------------------------------------------------------------
\documentclass[load-preamble+]{cnltx-doc}
\usepackage{sepfootnotes,fnpct}
\setcnltx{
  package = fnpct ,
  authors = Clemens Niederberger ,
  email   = contact@mychemistry.eu ,
  url     = https://github.com/cgnieder/fnpct/ ,
  abstract = {
      The \fnpct\ package basically does two things to footnotes: if footnote
      marks are followed by a punctuation mark the order of footnote and
      punctuation mark is reversed and the kerning gets adjusted.  As a side
      effect a new method of creating multiple footnotes and some other
      features are provided.%
  },
  add-cmds = {
    AdaptNote,
    AdaptNoteMark,
    innernote,
    kfc,
    kfp,
    multfootnote,
    setfnpct,
    writeinnernotes
  } ,
  add-silent-cmds = {
    anote,
    anotecontent,
    arabic,
    DeclareNewFootnote,
    endnote,
    footnoteB,
    multendnote,
    multfootcite,
    multparnote,
    multsidenote,
    myfn,
    newfootnotes,
    parnote,
    parnotes,
    sidenote,
    theendnotes,
    thempfootnote,
  },
  makeindex-setup = {options={-s cnltx.ist},columns=3,columnsep=1em} ,
  index-setup = {othercode=\footnotesize,level=\addsec}
}

\defbibheading{bibliography}[\bibname]{\addsec{#1}}

\usepackage{array,booktabs}

\setfnpct{multiple,add-punct-marks=:[0pt][-.05em]}

\usepackage{endnotes}
\usepackage{parnotes}

\newfootnotes{a}

\renewcommand\thempfootnote{\arabic{mpfootnote}}

\addbibresource{\jobname.bib}
\addbibresource{biblatex-examples.bib}
\usepackage{filecontents}
\begin{filecontents*}{\jobname.bib}
@book{bringhurst04,
  title     = {The Elements of Typographic Style},
  author    = {Robert Bringhurst},
  year      = {2004},
  version   = {3.2},
  isbn      = {978-0-88179-205-5},
  publisher = {Hartley \&\ Marks, Canada}
}
\end{filecontents*}

\AdaptNoteOpt\footcite\multfootcite

\newname\bringhurst{Robert Bringhurst}
\newname\lefloch{Bruno Le Floch}

\begin{document}

\section{License and Requirements}
\license\

\fnpct\ depends on the packages \bnd{l3kernel}~\cite{bnd:l3kernel},
\pkg{xparse} and \pkg{l3keys2e} from the \bnd{l3packages}
bundle~\cite{bnd:l3packages}, \pkg{scrlfile} from the \KOMAScript\
bundle~\cite{bnd:koma-script}, and \pkg{translations}~\cite{pkg:translations}.

\section{What's it all about?}
\subsection{Introduction}
The purpose of this package is to offer kerning for footnote marks, \ie\ the
superscripts. This is not appropriate for all superscripts.  Symbols must be
handled differently than numbers.  And of course the amount depends on the
chosen font.  \bringhurst\ says in \citetitle{bringhurst04}:

\begin{cnltxquote}[\emph{\citetitle{bringhurst04}} \cite{bringhurst04}]
  Superscripts frequently come at the ends of phrases or sentences.  If they
  are high above the line, they can be kerned over a comma or period, but this
  may endanger readability, especially if the text is set in a modest size.
\end{cnltxquote}

\fnpct\ can not make these decisions for you.  It sets some initial values for
the comma and the full stop which looked good to me with the tested fonts.
Additionally it kerns the superscripts away from words when it follows
directly.  The amounts of the kerning can be changed using an option.

As a side effect \fnpct\ switches the order of the superscript and the
following comma or full stop.  Additional punctuation marks can be added to
the switching behaviour and the amount of kerning can be set for each
punctuation mark individually.

\subsection{Basics}
The \fnpct\ package basically does two things to footnotes: if footnote marks
are followed by a comma or a full stop\footnote{More punctuation marks can be
  added through a package option.} the order of footnote and punctuation mark
is reversed and the kerning gets adjusted.  As a side effect a new method of
creating multiple footnotes\footnote{like;this} is provided, see
section~\ref{sec:multiple} for details.

In what way is the kerning adjusted?  After being placed behind the
punctuation mark the footnote mark is moved a little bit back, namely by the
amount specified with the option \option{after-punct-space} (see
section~\ref{sec:options}).  If the footnote mark follows a word
\emph{without} being followed by a punctuation\footnote{Well, it does not
  necessarily have to follow a \emph{word}.  The important point is \emph{not
    being followed} by a punctuation mark.}, there (obviously) is no order
switching and a little space is inserted before the footnote mark, namely the
amount specified by the option \option{before-footnote-space} (see
section~\ref{sec:options}).

\begin{sourcecode}
  \renewcommand\thempfootnote{\arabic{mpfootnote}}
\end{sourcecode}

Now, let's see some action:
\begin{example}
  \begin{minipage}{.4\linewidth}
    \noindent The three little pigs built their houses
    out of straw\footnote{not to be confused with hay},
    sticks\footnote{or lumber according to some sources}
    and bricks\footnote{probably fired clay bricks}.
  \end{minipage}
\end{example}

To ensure that the kerning is set the right way the footnote \emph{must} be
placed \emph{before} the full stop or the comma. \emph{The command can look
  ahead but not look back}.  This means if you place the \cs{footnote} command
after a full stop or a comma it is treated as if following a word, \ie\ a thin
space is inserted: effectively the opposite of the desired behaviour.

The order-switching can be prevented using a package option since not all
countries and languages have the same typographic conventions. In this case
the full stop and the comma are moved a bit back.

\begin{figure}[ht]
  \centering
  \begin{tabular}{>{\setfnpct{dont-mess-around}}ll}
    \toprule
      without \fnpct &
      with \fnpct \\
    \midrule
      \strut\quad text.\footnotemark[1] &
      \strut\quad text\footnotemark[1]. \\
      \strut\quad text\footnotemark[1]. &
      \setfnpct{punct-after}\strut\quad text\footnotemark[1]. \\
      \strut\quad text\footnotemark[1] &
      \strut\quad text\footnotemark[1] \\
    \bottomrule
  \end{tabular}
  \caption{The effect of \fnpct.}
\end{figure}

\begin{options}
  \keybool{punct-after}\Default{false}
    When \code{true} the punctuation sign will be placed \emph{after} the
    footnote.
\end{options}
Like all options it can also be set using the setup command.
\begin{commands}
  \command{setfnpct}[\marg{options}]
    Set up options.  Can be used anywhere in the document.  Some options can
    only be set in the preamble, though.
\end{commands}
\begin{example}
  \setfnpct{punct-after}
    \begin{minipage}{\linewidth}
    \noindent The three little pigs built
    their houses out of straw\footnote{not
      to be confused with hay},  sticks%
    \footnote{or lumber according to some
      sources} and bricks\footnote{probably
      fired clay bricks}.
  \end{minipage}
\end{example}

\subsection{Temporarily disable or enable switching}
One maybe want to put some footnote marks \emph{before} the punctuation and
some after, for example because the first one describes a single word but the
second one a whole sentence.  For this purpose \fnpct\ adds a \sarg\ argument
to \cs{footnote} and \cs{footnotemark}.  The complete new syntax now is as
follows:
\begin{commands}
  \command{footnote}[\sarg\oarg{num>}\marg{footnote text}\meta{tpunct}]
    new \sarg\ argument added.  \meta{tpunct} is the optional trailing
    punctuation mark.
  \command{footnotemark}[\sarg\oarg{num>}\meta{tpunct}]
    new \sarg\ argument added.  \meta{tpunct} is the optional trailing
    punctuation mark.
\end{commands}
The \sarg\ argument temporarily turns off the punctuation/footnote switching.
In case you set \keyis{punct-after}{true} the \sarg\ argument temporarily
\emph{enables} the switching.
\begin{example}
  \begin{minipage}{.4\linewidth}
    \noindent The three little pigs built their houses
    out of straw\footnote*{not to be confused with hay},
    sticks\footnote{or lumber according to some sources}
    and bricks\footnote{probably fired clay bricks}.
  \end{minipage}\hfil
  \setfnpct{punct-after}
  \begin{minipage}{.4\linewidth}
    \noindent The three little pigs built their houses
    out of straw\footnote*{not to be confused with hay},
    sticks\footnote{or lumber according to some sources}
    and bricks\footnote{probably fired clay bricks}.
  \end{minipage}
\end{example}

\subsection{Language Specific Settings}
Package \pkg{babel}'s~\cite{pkg:babel} French settings redefine
\cs*{@footnotemark}.  \fnpct\ resets this redefinition but sets the
\option{before-footnote-space} equal to a thin space (\code{.16667em}).

\subsection{Changes in v0.5}\label{ssec:news}

\begin{itemize}
  \item the option \option{bigfoot-default-top} has been removed
  \item the \pkg{manyfoot} package is now supported
\end{itemize}

\section{Options}\label{sec:options}
Most package options are listed below.  They all can be set as a package
option or with the \cs{setfnpct} command.  Most of them are for adjusting the
kerning.

\default{Underlined} values are set if the option is used without value.
\begin{options}
  \keychoice{add-punct-marks}{\meta{mark}\oarg{space after}\oarg{space before}}
    Add another punctuation mark for the position switching and spacify
    kerning if necessary.  An existing entry will be overwritten.  This option
    accepts a list of marks.
  %
  \keyval{remove-punct-marks}{mark}
    Remove punctuation mark from the kerning/position switching behaviour.
    This option accepts a list of marks.
  %
  \keyval{after-dot-space}{dim}\Default{-.06em}
    Space to be inserted after a footnote mark and before the full stop
    (``footnote \emph{after} dot'').  A negative space will move the full stop
    further to the footnote mark.
  %
  \keyval{after-comma-space}{dim}\Default{-.06em}
    Space to be inserted after a footnote mark and before the comma
    (``footnote \emph{after} comma'').  A negative space will move the comma
    further to the footnote mark.
  %
  \keyval{after-punct-space}{dim}
    Set \emph{all} spaces, \ie\ also the ones for marks you've added with
    \option{add-punct-marks}.
  %
  \keybool{punct-after}\Default{false}
    When \code{true} the punctuation sign will be placed \emph{after} the
    footnote.
  %
  \keyval{before-dot-space}{dim}\Default{-.16em}
    Space to be inserted after a footnote mark and before the full stop
    (``footnote \emph{before} dot''), \ie\ with \keyis{punct-after}{true}.
  %
  \keyval{before-comma-space}{dim}\Default{-.16em}
    space to be inserted after a footnote mark and before the comma
    (``footnote \emph{before} comma''), \ie\ with \keyis{punct-after}{true}.
  %
  \keyval{before-punct-space}{dim}
    Set \emph{all} spaces, \ie\ also the ones for marks you've added with
    \option{add-punct-marks}.
  %
  \keyval{before-footnote-space}{dim}\Default{.06em}
    Space to be inserted between a word and the following footnote mark.
  %
  \keybool{dont-mess-around}\Default{false}
    Sets all mentioned lengths to \code{0pt} and \keyis{punct-after}{true}.
    Basically this should look like as if you hadn't loaded \fnpct\ except the
    multiple footnote commands and \cs{innernote} are still available, see
    sections~\ref{sec:multiple} and~\ref{sec:nested}, respectively.
  %
  \keybool{multiple}\Default{false}
    Lets \cs{footnote} be equal to \cs{multfootnote}.
  %
  \keyval{mult-fn-delim}{delimiter}\Default{;}
    Sets the delimiter for the \cs{multfootnote} command.
  %
  \keyval{mult-fn-sep}{separator}\Default{,}
    Sets the separator between multiple footnote marks.
  %
  \keybool{normal-marks}\Default{false}\label{opt:normal-marks}
    Sets the footnote marks in the foot not as superscripts but as normal
    font.  If you're not using a \KOMAScript\ class this option will load
    the package \pkg{scrextend}.  See section~\ref{sec:normal} for more
    information.  \emph{This option can only be set in the preamble}.
\end{options}

So if you want to extend the punctuation switching and kerning to other punctuation
marks you can do something like this:
\begin{example}
  \setfnpct{add-punct-marks=!?}
  \begin{minipage}{.4\linewidth}
    \noindent The three little pigs built their houses
    out of straw\footnote{not to be confused with hay}?
    Sticks\footnote{or lumber according to some sources}
    and bricks\footnote{probably fired clay bricks}!
  \end{minipage}
\end{example}

Or with some additional spacing:
\begin{example}
  \setfnpct{add-punct-marks=![.03em]?[.03em]}
  \begin{minipage}{.4\linewidth}
    \noindent The three little pigs built their houses
    out of straw\footnote{not to be confused with hay}?
    Sticks\footnote{or lumber according to some sources}
    and bricks\footnote{probably fired clay bricks}!
  \end{minipage}
\end{example}

Let's take a look at an example with some ridiculous settings:
\begin{example}
  % some ridiculous settings:
  \setfnpct{after-punct-space=2pt,before-footnote-space=2pt}
  \begin{minipage}{.4\linewidth}
    \noindent The three little pigs built their houses
    out of straw\footnote{not to be confused with hay},
    sticks\footnote{or lumber according to some sources}
    and bricks\footnote{probably fired clay bricks}.
  \end{minipage}
\end{example}

And now the same with switched order:
\begin{example}
  % some ridiculous settings:
  \setfnpct{punct-after,before-punct-space=2pt,before-footnote-space=2pt}
  \begin{minipage}{.4\linewidth}
    \noindent The three little pigs built their houses
    out of straw\footnote{not to be confused with hay},
    sticks\footnote{or lumber according to some sources}
    and bricks\footnote{probably fired clay bricks}.
  \end{minipage}
\end{example}

In case the automatic kerning doesn't work for whatever reason the spaces are
available as user commands as well:
\begin{commands}
  \command{kfp}
    Insert \option{after-dot-space} or (if \keyis{punct-after}{true})
    \option{before-dot-space}. 
  \command{kfc}
    Insert \option{after-comma-space} or (if \keyis{punct-after}{true})
    \option{before-comma-space}.
\end{commands}

Some of the options are explained in a bit more detail in the next sections.

\section{Multiple footnotes}\label{sec:multiple}
\subsection{Basics}
Since multiple footnotes have to be treated differently\footnote{see these
  footnotes;for an example}, \fnpct\ provides an extra command for that:
\begin{commands}
  \command{multfootnote}[\sarg\Marg{\meta{list};\meta{of};\meta{footnotes}}\meta{tpunct}]
    Different footnotes are separated with a semicolon.  The \sarg\ turns the
    footnote/punctuation switching temporarily off.
\end{commands}
\begin{example}
  \begin{minipage}{.4\linewidth}
    \noindent The three little pigs built their houses
    out of straw\footnote{not to be confused with hay},
    sticks\footnote{or lumber according to some sources}
    and bricks\multfootnote{probably fired clay bricks;or
    something else}.
  \end{minipage}
\end{example}

Every of the items of the list has an optional argument equivalent to the
optional argument of \cs{footnotetext}:
\begin{commands}
  \command{multfootnote}[\Marg{\sarg\oarg{mark}\meta{footnote text};\ldots}]
    The syntax of \cs{multfootnote}'s footnote text argument.
\end{commands}
\begin{example}
  \begin{minipage}{.4\linewidth}
    \noindent The three little pigs built their houses
    out of straw\footnote{not to be confused with hay},
    sticks\footnote{or lumber according to some sources}
    and bricks\multfootnote{probably fired clay bricks;[5]or
    something else}.
  \end{minipage}
\end{example}

Additionally every item has an optional \sarg\ which only invokes
\cs{footnotetext}.  This enables for example to set nested footnotes without
disrupting the multiple setting.  The following example is shown in
figure~\ref{fig:nested:mult}:
\begin{example}[compile,float,caption={Nested footnotes.\label{fig:nested:mult}}]
  \documentclass{article}
  \usepackage[
    paperwidth=.5\textwidth,
    paperheight=12\baselineskip,
    margin=5pt,
    bottom=1.5cm]{geometry}
  
  \usepackage{fnpct}
  
  \begin{document}
  \noindent The three little pigs built their houses
  out of straw\footnote{not to be confused with hay},
  sticks\footnote{or lumber according to some sources}
  and bricks\multfootnote{probably fired clay bricks%
  \footnotemark;*or something else;what do I know}.
  \end{document}
\end{example}

\subsection{The delimiter and separator}
Since the semicolon might be part of the footnote text you might have some
trouble.  But there are ways around. Maybe try enclosing it in braces:
\begin{example}
  \begin{minipage}{.4\linewidth}
    \noindent The three little pigs built their houses
    out of straw\footnote{not to be confused with hay},
    sticks\footnote{or lumber according to some sources}
    and bricks\multfootnote{probably fired clay bricks{;}
    or something else;what do I know}.
  \end{minipage}
\end{example}

There also are options which lets you choose the (input) delimiter and the (output)
separator:
\begin{options}
  \keyval{mult-fn-delim}{delimiter}\Default{;}
    Choose delimiter for the \cs{multfootnote}.
  \keyval{mult-fn-sep}{separator}\Default{,}
    Choose the separator that is put between footnote marks.
\end{options}
\begin{example}
  \setfnpct{mult-fn-delim=//,mult-fn-sep=;}
  \begin{minipage}[t]{.4\linewidth}
    \noindent The three little pigs built their houses
    out of straw\footnote{not to be confused with hay},
    sticks\footnote{or lumber according to some sources}
    and bricks\multfootnote{probably fired clay bricks;
    or something else//what do I know}.
  \end{minipage}\hfil
  \setfnpct{mult-fn-delim=;,mult-fn-sep=}
  \renewcommand*\thempfootnote{\fnsymbol{mpfootnote}}
  \begin{minipage}[t]{.4\linewidth}
    \noindent The three little pigs built their houses
    out of straw\footnote{not to be confused with hay},
    sticks\footnote{or lumber according to some sources}
    and bricks\multfootnote{probably fired clay bricks;
    or something else}.
  \end{minipage}
\end{example}

\subsection{Automagic}
If you want you can turn all footnotes into \cs{multfootnote}s.
\begin{options}
  \keybool{multiple}\Default{false}
    Let \cs{footnote} behave like \cs{multfootnote}.  \emph{This option can
      only be set in the preamble}.
\end{options}
\begin{example}
  % in preamble: \setfnpct{multiple}
  % or \usepackage[multiple]{fnpct}
  \begin{minipage}{.4\linewidth}
    \noindent The three little pigs built their houses
    out of straw\footnote{not to be confused with hay},
    sticks\footnote{or lumber according to some sources}
    and bricks\footnote{probably fired clay bricks;or
      something else}.
  \end{minipage}
\end{example}

\section{Nested Footnotes}\label{sec:nested}
\fnpct\ provides preliminary support for nested footnotes.  There are some
restrictions, though.  The command below can be used inside \cs{footnote} and
\cs{multfootnote}, but not inside another \cs{innernote}.  \cs{innernote}
really is a ``inner footnote'', it doesn't work with endnotes, for example.
\begin{commands}
  \command{innernote}[\sarg\oarg{mark}\marg{text}\meta{tpunct}]
    A nested footnote.
\end{commands}
The inner node uses a \cs{footnotemark} internally and the mark will be set
with the definition it has at begin document.

The output of the following code can be seen in figure~\ref{fig:nested:a}.
\begin{example}[compile,float,caption={\cs*{innernote} example (a)\label{fig:nested:a}}]
  \documentclass{article}
  \usepackage[
    paperwidth=.5\textwidth,
    paperheight=12\baselineskip,
    margin=5pt,
    bottom=1.5cm]{geometry}
  
  \usepackage{fnpct}
  
  \begin{document}
  \noindent The three little pigs built their houses
  out of straw\footnote{not to be confused with hay},
  sticks\footnote{or lumber according to some sources}
  and bricks\footnote{probably fired clay bricks%
    \innernote{or not}}.
  \end{document}
\end{example}

This is especially useful if you have more than one additional footnote inside
a footnote, a case that can't be solved in a satisfying (\ie\ automatic) way
with the method mentioned in section~\ref{sec:multiple}.  See
figure~\ref{fig:nested:b} for the output of this code:
\begin{example}[compile,float,caption={\cs*{innernote} example (b)\label{fig:nested:b}}]
  \documentclass{article}
  \usepackage[
    paperwidth=.5\textwidth,
    paperheight=12\baselineskip,
    margin=5pt,
    bottom=1.5cm]{geometry}
  
  \usepackage{fnpct}
  
  \begin{document}
  \noindent The three little pigs built their houses
  out of straw\footnote{not to be confused with hay},
  sticks\footnote{or lumber according to some sources}
  and bricks\footnote{probably\innernote{just guessing}
    fired clay bricks\innernote{or not}}.
  \end{document}
\end{example}

\cs{innernote} can also be used to escape minipages.  In this case use
\cs{innernote} as a replacement for \cs{footnote} and call this command later:
\begin{commands}
  \command{writeinnernotes}
    Writes all \cs{footnotetext}s to the inner footnotes when invoked.  This
    is usually done automatically by the \cs{footnote} command but there are
    possible needs for it.
\end{commands}
See figure~\ref{fig:nested:c} for the output of the following code:
\begin{example}[compile,float,caption={\cs*{innernote} example (c)\label{fig:nested:c}}]
  \documentclass{article}
  \usepackage[
    paperwidth=.5\textwidth,
    paperheight=12\baselineskip,
    margin=5pt,
    bottom=1.5cm]{geometry}

  \usepackage{fnpct}

  \begin{document}
  \noindent\fbox{%
  \begin{minipage}{.9\linewidth}
     The three little pigs built their houses
     out of straw\innernote{not to be confused with hay},
     sticks\innernote{or lumber according to some sources}
     and bricks\innernote{probably}.
  \end{minipage}}\writeinnernotes
  \end{document}
\end{example}

\section{Normal notes}\label{sec:normal}
The package option \option{normal-marks} (see p.\,\pageref{opt:normal-marks})
changes the layout of the footnote marks in the foot by using \KOMAScript's
\cs{deffootnote} command:
\begin{commands}
  \command{deffootnote}[\oarg{mark indent}\marg{indent}\marg{par
    indent}\marg{format code}]
    \KOMAScript's command to define the output of the footnote text in the
    foot.
\end{commands}

The following code is shown in figure~\ref{fig:normal}.
\begin{example}[compile,float,caption={Demonstration of the
    \option*{normal-marks} option.\label{fig:normal}}]
  \documentclass{article}
  \usepackage[
    paperwidth=.5\textwidth,
    paperheight=12\baselineskip,
    margin=5pt,
    bottom=1.5cm]{geometry}
  
  \usepackage[normal-marks]{fnpct}
  
  \begin{document}
  
  \noindent The three little pigs built their houses
  out of straw\footnote{not to be confused with hay},
  sticks\footnote{or lumber according to some sources}
  and bricks\footnote{probably fired clay bricks}.
  
  \end{document}
\end{example}

The three indent values used by \fnpct\ can be changed with the following
options:
\begin{options}
  \keyval{normal-mark-width}{dim}\Default{1em}
    The width of the box within which the marks are set.
  \keyval{normal-indent}{dim}\Default{1.5em}
    The indent of each footnote line \emph{except} the first.
  \keyval{normal-par-indent}{dim}\Default{1em}
    Additional indent of a new paragraph inside the footnote text.
\end{options}

\section{Other packages}\label{sec:other:packages}
\fnpct\ tries its best to support other footnote and related
packages\footnote{If you find some package missing please let me know.}.  Each
of the following subsections is dedicated to one of these packages and if and
how they work together with \fnpct.  Fortunately most of them do quite well.

\subsection{bigfoot and manyfoot}\label{ssec:bigfoot}
The \pkg{bigfoot}~\cite{pkg:bigfoot} package extends the possibilities of the
\pkg{manyfoot} package~\cite{pkg:manyfoot} (see section~\ref{ssec:manyfoot}).
The main visible feature is to use different classes of footnotes which are
typeset in different apparatus on the bottom of the page.

\fnpct\ is compatible with \pkg{bigfoot}.

You need to be a bit cautious, though: \emph{verbatim material won't work
  inside footnotes anymore}.  You can of course use \lefloch's
\pkg{cprotect}~\cite{pkg:cprotect} if you really need verbatim material in
footnotes \emph{and} want to use \fnpct.  Since \fnpct\ does not redefine any
\cs{footnotetext} like command it will still work inside one of them.

For every footnote class defined with \cs*{DeclareNewFootnote} the commands
\cs*{footnote\meta{class}} and \cs*{footnotemark\meta{class}} are redefined
with the starred variant and a \cs*{multfootnote\meta{class}} is defined. The
\option{multiple} option will turn all \cs*{footnote\meta{class}} commands
into the corresponding \cs*{multfootnote\meta{class}}.

Since this document cannot easily combine \emph{every} footnote package at the
same time for demonstration purposes the following code is shown in
figure~\ref{fig:bigfoot}:
\begin{example}[compile,float,caption={\pkg*{bigfoot} example\label{fig:bigfoot}}]
  \documentclass{article}
  \usepackage[
    paperwidth=.5\textwidth,
    paperheight=12\baselineskip,
    margin=5pt,
    bottom=1.5cm]{geometry}
  
  \usepackage{bigfoot}
  \usepackage{fnpct}
  \setfnpct{multiple}
  \DeclareNewFootnote{default}
  \DeclareNewFootnote[para]{B}[alph]
  
  \begin{document}
  
  \noindent The three little pigs built their houses
  out of straw\footnote*{not to be confused with hay%
  \footnoteB{let alone grass}}, sticks\footnote{or
    lumber according to some sources} and bricks%
  \footnote{probably fired clay bricks;or something}.
  
  \end{document}
\end{example}

\subsection{endnotes}
The \pkg{endnotes}~\cite{pkg:endnotes} package povides the commands
\cs{endnote} and \cs{endnotemark} which can be used to output all notes at the
end of a chapter, say.

If the package is loaded both commands are adapted\footnote*{see
  section~\ref{sec:adapt} on manual adaption}:
\begin{sourcecode}
  % `fnpct' does this for you:
  \AdaptNote\endnote\multendnote
  \AdaptNoteMark\endnotemark
\end{sourcecode}

Everything else stays the same:
\begin{example}
  \begin{minipage}[t]{.4\linewidth}
    \noindent The three little pigs built their houses
    out of straw\endnote{not to be confused with hay},
    sticks\endnote{or lumber according to some sources}
    and bricks\endnote{probably fired clay bricks}.
  
    \theendnotes
  \end{minipage}\hfil
  \begin{minipage}[t]{.4\linewidth}
    \noindent The three little pigs built their houses
    out of straw\endnote*{not to be confused with hay},
    sticks\endnote{or lumber according to some sources}
    and bricks\multendnote{probably fired clay bricks;or
    something else}.
    
    \theendnotes
  \end{minipage}
\end{example}

\subsection{enotez}
The \pkg{enotez}~\cite{pkg:enotez} package povides the command \cs{endnote}
which can be used to output all notes at the end of a chapter, say.

If the package is loaded the command is adapted\footnote*{see
  section~\ref{sec:adapt} on manual adaption}:
\begin{sourcecode}
  % `fnpct' does this for you:
  \AdaptNote\endnote\multendnote
\end{sourcecode}

\subsection{fixfoot}
The \pkg{fixfoot}~\cite{pkg:fixfoot} package provides a possibility to create
repeating footnotes.  \fnpct\ supports this package provided you take care of
the following:

\emph{Use \cs{DeclareFixedFootnote} only in the preamble but \emph{after}
  loading \fnpct.}

\cs*{DeclareFixedFootnote}\Marg{\cs*{cs}}\marg{footnote text} is used to store
the \meta{footnote text} in \cs*{cs} which in turn creates a footnote mark for
it but doesn't repeat the footnote text on the same page in the bottom.  The
document needs several runs to get all the numbers and footnotes right.

Every fixed footnote declared with \cs{DeclareFixedFootnote} gets an optional
\sarg\ to prevent the punctuation switching.  \emph{There is \emph{no}
  mult-variant of the fixed notes!}

Since this document cannot easily combine \emph{every} footnote package at the
same time for demonstration purposes the following code is shown in
figure~\ref{fig:fixfoot}:
\begin{example}[compile,float,caption={\pkg*{fixfoot}
    example\label{fig:fixfoot}}]
  \documentclass{article}
  \usepackage[
    paperwidth=.55\textwidth,
    paperheight=12\baselineskip,
    margin=5pt,
    bottom=1.5cm]{geometry}
    
  \usepackage{fixfoot}
  \usepackage{fnpct}
  
  \DeclareFixedFootnote{\myfn}{I'm confused: what do I want to say?}
  
  \begin{document}
  
  \noindent The three little pigs\myfn\ built their houses
  out of straw\myfn*, sticks\footnote{or lumber according
  to some sources} and bricks\footnote{probably fired clay
    bricks}.
  
  \end{document}
\end{example}

\subsection{footmisc}\label{ssec:footmisc}
The \pkg{footmisc} package~\cite{pkg:footmisc} provides a range of options to
customize footnotes, for example output them as margin notes or count
footnotes per page.

Testing showed no incompatibilities with \pkg{footmisc}.  The only thing is
that you won't have to (and shouldn't) use its \option{multiple} option.

\subsection{footnote}\label{ssec:footnote}
Unfortunately \fnpct\ is not compatible with the \pkg{footnote}
package~\cite{pkg:footnote}.  Or more precisely it is not compatible if the
command pair \cs*{savenotes}/\cs*{spewnotes} is invoked.  This disables
\pkg{footnote}'s environments as well as its environment escaping mechanism
through \cs{makesavenoteenv}.

\subsection{manyfoot}\label{ssec:manyfoot}
The package \pkg{manyfoot}~\cite{pkg:manyfoot} is supported.  See
section~\ref{ssec:bigfoot} for more details.

\subsection{pagenote}\label{ssec:pagenote}
The package \pkg{pagenote}~\cite{pkg:pagenote} is supported and its
\cs*{pagenote} command is treated the same way as all other note commands: it
got an optional \sarg\ argument and an additional \cs*{multpagenote} is
defined.

\subsection{parnotes}\label{ssec:parnotes}
The function of the package \pkg{parnotes}~\cite{pkg:parnotes} is similar to the
\pkg{endnotes} package.  basically it allows to output the footnote text after
a paragraph, either by using a special environment or by invoking
\cs*{parnotes}.

If \pkg{parnotes} is loaded \fnpct\ extends its functionality analogous to the
one of \pkg{endnotes}.

Again the \option{multiple} option turns all \cs*{parnote}s into
\cs*{multparnote}s.
\begin{example}
  \begin{minipage}[t]{.4\linewidth}
    \noindent The three little pigs built their houses
    out of straw\parnote{not to be confused with hay},
    sticks\parnote{or lumber according to some sources}
    and bricks\parnote{probably fired clay bricks}.
    
    \parnotes
  \end{minipage}\hfil
  \begin{minipage}[t]{.4\linewidth}
    \noindent The three little pigs built their houses
    out of straw\parnote*{not to be confused with hay},
    sticks\parnote{or lumber according to some sources}
    and bricks\multparnote{probably fired clay bricks;or
      something else}.
    
    \parnotes
  \end{minipage}
\end{example}

\subsection{sepfootnotes}\label{ssec:sepfootnotes}
\fnpct\ supports the \pkg{sepfootnotes} package~\cite{pkg:sepfootnotes}.  Each
note type you have created with with one of the commands
\begin{description}
  \item \cs*{newfootnotes}\marg{prefix},
  \item \cs*{newendnotes}\marg{prefix} or
  \item \cs*{newsymbolfootnotes}\marg{prefix}
\end{description}
gets the additional optional \sarg\ to prevent punctuation switching.  Also
the corresponding mult-variant \cs*{\meta{prefix}multnote} is defined.  The
option \option{multiple} again turns all note commands into their
mult-variant.

You have to use the \cs*{newfootnotes} commands in the document preamble after
loading \fnpct.

\begin{example}
  % preamble \newfootnotes{a}
  \anotecontent{first}{not to be confused with hay}
  \anotecontent{second}{or lumber according to some sources}
  \anotecontent{third}{probably fired clay bricks}
  \begin{minipage}{.4\linewidth}
    \noindent The three little pigs built their houses
    out of straw\anote{first}, sticks\anote{second} and
    bricks\anote{third}.
  \end{minipage}
\end{example}

\subsection{sidenotes}\label{ssec:sidenotes}
The package \pkg{sidenotes}~\cite{pkg:sidenotes} is \emph{not} supported.
For the time being users can use the following code\footnote{Since the code
  depends on the implementation of \pkg{sidenotes} this may break with each
  update of \pkg{sidenotes}.;Thanks to Marijn van Vliet for
  providing it!}:
\begin{sourcecode}
  \ExplSyntaxOn
  \makeatletter
  \NewDocumentCommand \origsidenotemark {o}
    {
      \@sidenotes@multichecker
      \IfNoValueOrEmptyTF {#1}
        { \@sidenotes@thesidenotemark { \thesidenote } }
        { \@sidenotes@thesidenotemark {#1} }
      \@sidenotes@multimarker
    }

  \NewDocumentCommand \origsidenote {oo+m}
    {
      \origsidenotemark [#1]
      \sidenotetext [#1] [#2] {#3}
      \@sidenotes@multimarker
    }

  \fnpct_create_mult_variant:NN \multsidenote \origsidenote

  \RenewDocumentCommand \sidenote {soo+m}
    { \fnpct_handle_note:Nnnnn \origsidenote {#2} {#3} {#4} {#1} }

  \RenewDocumentCommand \sidenotemark {so}
    { \fnpct_handle_note:Nnnnn \origsidenotemark {#2} { \q_no_value } { } {#1} }
  \makeatother
  \ExplSyntaxOff
\end{sourcecode}

\subsection{snotez}\label{ssec:snotez}
The package \pkg{snotez}~\cite{pkg:snotez} is supported and its commands
\cs*{sidenote} and \cs*{sidenotemark} are treated the same way as footnotes:
they got an optional \sarg\ argument and the additional \cs*{multsidenote} is
created.  The option \option{multiple} turns all note commands into their
mult-variant.

\subsection{tablefootnote}\label{ssec:tablefootnote}
The package \pkg{tablefootnote}~\cite{pkg:tablefootnote} is supported and its
\cs*{tablefootnote} command is treated the same way as all other note
commands: it got an optional \sarg\ argument and an additional
\cs*{multtablefootnote} is defined.  The option \option{multiple} turns all
note commands into their mult-variant.

\subsection{yafoot}\label{ssec:yafoot}
Currently no issues are known when using \fnpct\ with one or all of the
packages of the \bnd{yafoot}~\cite{bnd:yafoot} bundle.  Just for reference:
these packages are \pkg*{pfnote}, \pkg*{fnpos} and \pkg*{dblfnote}.

\section{Adapting note commands}\label{sec:adapt}
If a package is not natively supported you can try and adapt commands yourself.
You'll need to use one of the following commands. Please be aware that if the
adapted command does not match the required syntax it will get the required syntax
afterwards. This can result in dummy arguments or an overwritten \code{*} argument.

\begin{commands}
  % \AdaptNote
  \command{AdaptNote}[\Marg{\cs*{cs}}\Marg{\cs*{multcs}}]
    This lets \cs*{cs}  adapt to the kerning behaviour and adds an optional
    \sarg.  It also creates a corresponding mult-variant.  The package option
    \option{multiple} will now turn \cs*{cs} into its mult-variant.  Required
    syntax: \cs*{cs}\oarg{oarg}\marg{marg}
  % \AdaptNoteNoMult
  \command{AdaptNoteNoMult}[\Marg{\cs*{cs}}]
    Like \cs{AdaptNote} but doesn't create a mult-variant and thus there is no
    change in behaviour when the option \option{multiple} is used.  Required
    syntax: \cs*{cs}\oarg{oarg}\marg{marg}
  % \AdaptNoteOpt
  \command{AdaptNoteOpt}[\Marg{\cs*{cs}}\Marg{\cs*{multcs}}]
    This lets \cs*{cs} adapt to the kerning behaviour and adds an optional
    \sarg.  It also creates a corresponding mult-variant.  The package option
    \option{multiple} will now turn \cs*{cs} into its mult-variant.  Required
    syntax: \cs*{cs}\oarg{oarg}\oarg{oarg}\marg{marg}
  % \AdaptNoteOptNoMult
  \command{AdaptNoteOptNoMult}[\Marg{\cs*{cs}}]
    Like \cs{AdaptNoteOpt} but doesn't create a mult-variant and thus there is
    no change in behaviour when the option \option{multiple} is used.
    Required syntax: \cs*{cs}\oarg{oarg}\oarg{oarg}\marg{marg}
  % \AdaptNoteNoOpt
  \command{AdaptNoteNoOpt}[\Marg{\cs*{cs}}\Marg{\cs*{multcs}}]
    \sinceversion{0.3}Like \cs{AdaptNote} but requires \cs*{cs} not to have an
    optional argument.  If it had one it wouldn't have it any more after the
    adaption.  The package option \option{multiple} will now turn \cs*{cs}
    into its mult-variant.  Required syntax: \cs*{cs}\marg{marg}
  % \AdaptNoteNoOptNoMult
  \command{AdaptNoteNoOptNoMult}[\Marg{\cs*{cs}}]
    \sinceversion{0.3}Like \cs{AdaptNoteNoOpt} but doesn't create a
    mult-variant.  Required syntax: \cs*{cs}\Marg{marg}
  % \AdaptNoteMark
  \command{AdaptNoteMark}[\Marg{\cs*{cs}}]
    This lets \cs*{cs} adapt to the kerning behaviour and adds an optional
    \sarg.  Required syntax: \cs*{cs}\oarg{oarg}
  % strict
\end{commands}
\begin{options}
  \keybool{strict}\Default{false} When \code{true}
    \cs{AdaptNote} (and the others) will issue errors if the note command in
    question has already been adapted.
\end{options}

\subsection{tufte-latex's side notes}
For example using the \cs{sidenote} command from the \cls*{tufte-handout}
class is easily possible, see also figure~\ref{fig:tufte}:
\begin{example}[compile,float,caption={Adapt
    \bnd*{tufte-latex}.\label{fig:tufte}},graphics={trim=2.5cm 24.5cm 2cm
    1cm,scale=.65}]
  \documentclass{tufte-handout}
  \usepackage[multiple]{fnpct}
  \AdaptNote\sidenote\multsidenote
  \begin{document}
  Some text in the document body\sidenote{Some
  text in a sidenote.;A second sidenote.}.
  \end{document}
\end{example}

\subsection{Adapting \cs*{footcite} and siblings}\label{ssec:adapt:footcite}
If you're willing to sacrifice \cs*{footcite}'s \sarg\ argument (which it has
in some styles) you can adapt the commands provided by
\pkg{biblatex}~\cite{pkg:biblatex} as well:

\begin{example}
  % used in preamble: \AdaptNoteOpt\footcite\multfootcite
  \begin{minipage}{.5\linewidth}
    \noindent The three little pigs built their houses
    out of straw\footnote{not to be confused with hay},
    sticks\footnote{or lumber according to some sources}
    and bricks\footnote{probably fired clay bricks}. The
    companion has nothing on this topic\multfootcite{[see]
      [but there's nothing there]companion;[or see][but there's
      also nothing]knuth:ct:a}.
  \end{minipage}
\end{example}

The option \option{multiple} turns all \cs*{footcite}s into
\cs*{multfootcite}s now.  Remember that the optional arguments then are inside
the braces!

\begin{example}
  % used in preamble: \AdaptNoteOpt\footcite\multfootcite
  % with option `multiple':
  \begin{minipage}{.5\linewidth}
    \noindent The three little pigs built their houses
    out of straw\footnote{not to be confused with hay},
    sticks\footnote{or lumber according to some sources}
    and bricks\footnote{probably fired clay bricks}. The
    companion has nothing on this topic\footcite{[see][but
      there's nothing there]companion;[or see][but there's also
      nothing]knuth:ct:a}.
  \end{minipage}
\end{example}


\appendix

\end{document}

% TODO: update documentation and explain what happens when babel's french is
% used
