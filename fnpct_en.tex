% --------------------------------------------------------------------------
% the FNPCT package v0.1alpha
% 
%   footnote kerning
% 
% 2012/05/18
% --------------------------------------------------------------------------
% Clemens Niederberger
% Web:    https://bitbucket.org/cgnieder/fnpct/
% E-Mail: contact@mychemistry.eu
% --------------------------------------------------------------------------
% Copyright 2012 Clemens Niederberger
% 
% This work may be distributed and/or modified under the
% conditions of the LaTeX Project Public License, either version 1.3
% of this license or (at your option) any later version.
% The latest version of this license is in
%   http://www.latex-project.org/lppl.txt
% and version 1.3 or later is part of all distributions of LaTeX
% version 2005/12/01 or later.
% 
% This work has the LPPL maintenance status `maintained'.
% 
% The Current Maintainer of this work is Clemens Niederberger.
% --------------------------------------------------------------------------
% The fnpct package consists of the files
%  - fnpct.sty,
%  - README
% --------------------------------------------------------------------------
% If you have any ideas, questions, suggestions or bugs to report, please
% feel free to contact me.
% --------------------------------------------------------------------------
% the package is inspired by the following question on TeX.SE:
%   http://tex.stackexchange.com/q/56094/5049
% --------------------------------------------------------------------------
% if you want to compile this documentation you'll need the document class
% `cnpkgdoc' which you can get here:
%    https://bitbucket.org/cgnieder/cnpkgdoc/
%    the class is licensed LPPL 1.3 or later
\documentclass[toc=index,toc=bib]{cnpkgdoc}
\docsetup{
  pkg      = fnpct ,
  code-box = {
    skipbelow = .5\baselineskip ,
    skipabove = .5\baselineskip ,
  }
}
\addcmds{arabic,kfc,kfp,multfootnote,setfnpct,thempfootnote}
\setfnpct{multiple}

\renewcommand\thempfootnote{\arabic{mpfootnote}}

\usepackage[backend=biber]{biblatex}
\addbibresource{biblatex-examples.bib}

\begin{filecontents}{fnpct_en.ist}
 preamble "\\begin{theindex}\n Section titles are indicated \\textbf{bold},
 packages \\textsf{sans serif}, commands \\code{\\textbackslash\\textcolor{code}{brown}}
 and options \\textcolor{key}{\\code{green}}.\\newline\n\n"
 heading_prefix "{\\bfseries "
 heading_suffix "\\hfil}\\nopagebreak\n"
 headings_flag  1
 delim_0 "\\dotfill "
 delim_1 "\\dotfill "
 delim_2 "\\dotfill "
 delim_r "\\textendash"
 suffix_2p "\\nohyperpage{\\,f.}"
 suffix_3p "\\nohyperpage{\\,ff.}"
\end{filecontents}

\makeindex
\begin{document}

\section{License and Requirements}
\fnpct is placed under the terms of the LaTeX Project Public License,
version 1.3 or later (\url{http://www.latex-project.org/lppl.txt}).
It has the status \enquote{maintained}.

\fnpct depends on the \paket{l3kernel} and \paket*{xparse} which is part of the
\paket{l3packages} bundle.

\section{What's it all about?}
The \fnpct package basically does two things to footnotes: if footnote marks are
followed by a comma or a dot the order of footnote and punctuation mark is reversed
and the kerning gets adjusted.

All examples in this documentation use
\begin{beispiel}[code only]
 \renewcommand\thempfootnote{\arabic{mpfootnote}}
\end{beispiel}

Now, let's see some action:
\begin{beispiel}
 \begin{minipage}{.4\linewidth}
  \noindent The three little pigs built their houses
  out of straw\footnote{not to be confused with hay},
  sticks\footnote{or lumber according to some sources}
  and bricks\footnote{probably fired clay bricks}.
 \end{minipage}
\end{beispiel}

The order-switching can be prevented using a package option
\begin{beschreibung}
 \option{punct-after}{\default{true}/false} when \code{true} the punctuation sign
 will be placed \emph{after} the footnote.
\end{beschreibung}
which like all options can also be set using the setup command.
\begin{beschreibung}
 \befehl{setfnpct}{<options>} set up options. Can be used anywhere in the document.
 Some options can only be set in the preamble, though.
\end{beschreibung}

\begin{beispiel}
 \setfnpct{punct-after}
 \begin{minipage}{.4\linewidth}
  \noindent The three little pigs built their houses
  out of straw\footnote{not to be confused with hay},
  sticks\footnote{or lumber according to some sources}
  and bricks\footnote{probably fired clay bricks}.
 \end{minipage}
\end{beispiel}

\achtung{Please be aware that this package is in an experimental state and there
hasn't been extensive testing with other footnote behaviour changing packages!}

\newpage
\section{Options}
Besides the \enquote{order switching} you can also adjust the kerning:
\begin{beschreibung}
 \option{after-dot-space}{<dim>} space to be inserted after a dot and before the
 footnote mark. Default = \code{-.06em}
 \option{after-comma-space}{<dim>} space to be inserted after a comma and before
 the footnote mark. Default = \code{-.06em}
 \option{after-punct-space}{<dim>} set both spaces.
 \option{punct-after}{\default{true}/false} when \code{true} the punctuation sign
 will be placed \emph{after} the footnote.
 \option{before-dot-space}{<dim>} space to be inserted before a dot and after the
 footnote mark, \ie with \key{punct-after}{true}. Default = \code{-.15em}
 \option{before-comma-space}{<dim>} space to be inserted before a comma and after
 the footnote mark, \ie with \key{punct-after}{true}. Default = \code{-.15em}
 \option{before-punct-space}{<dim>} set both spaces.
 \option{before-footnote-space}{<dim>} space to be inserted between a word and the
 following footnote mark. Default = \code{.06em}
 \option{multiple}{\default{true}/false} lets \cmd{footnote} be equal to
 \cmd{multfootnote} 
\end{beschreibung}

Let's take a look at an example:
\begin{beispiel}
 % some ridiculous settings:
 \setfnpct{after-punct-space=2pt,before-footnote-space=2pt}
 \begin{minipage}{.4\linewidth}
  \noindent The three little pigs built their houses
  out of straw\footnote{not to be confused with hay},
  sticks\footnote{or lumber according to some sources}
  and bricks\footnote{probably fired clay bricks}.
 \end{minipage}
\end{beispiel}

And now the same with switched order:
\begin{beispiel}
 % some ridiculous settings:
 \setfnpct{punct-after,before-punct-space=2pt,before-footnote-space=2pt}
 \begin{minipage}{.4\linewidth}
  \noindent The three little pigs built their houses
  out of straw\footnote{not to be confused with hay},
  sticks\footnote{or lumber according to some sources}
  and bricks\footnote{probably fired clay bricks}.
 \end{minipage}
\end{beispiel}

Some of the options are explained in a bit more detail in the next sections.

\section{Multiple footnotes}
\subsection{Basics}
Since multiple footnotes have to be treated differently\multfootnote{see this;as an example},
\fnpct provides an extra command for that:
\begin{beschreibung}
 \befehl{multfootnote}{<list>;<of>;<footnotes>} different footnotes are separated
 with a semicolon.
\end{beschreibung}
\begin{beispiel}
 \begin{minipage}{.4\linewidth}
  \noindent The three little pigs built their houses
  out of straw\footnote{not to be confused with hay},
  sticks\footnote{or lumber according to some sources}
  and bricks\multfootnote{probably fired clay bricks;or
  something else}.
 \end{minipage}
\end{beispiel}

Every of the items of the list has an optional argument equivalent to the optional
argument of \cmd{footnote}:
\begin{beispiel}
 \begin{minipage}{.4\linewidth}
  \noindent The three little pigs built their houses
  out of straw\footnote{not to be confused with hay},
  sticks\footnote{or lumber according to some sources}
  and bricks\multfootnote{probably fired clay bricks;[5]or
  something else}.
 \end{minipage}
\end{beispiel}

\subsection{The delimiter}
Since the semicolon might be part of the footnote text you might have some trouble.
But there are ways around. Maybe try enclosing it with braces:
\begin{beispiel}
 \begin{minipage}{.4\linewidth}
  \noindent The three little pigs built their houses
  out of straw\footnote{not to be confused with hay},
  sticks\footnote{or lumber according to some sources}
  and bricks\multfootnote{probably fired clay bricks{;}
  or something else;what do I know}.
 \end{minipage}
\end{beispiel}

There is an option which lets you choose the delimiter:
\begin{beschreibung}
 \option{mult-fn-delim}{<delimiter>} choose delimiter for the \cmd{multfootnote}.
 Default = \code{;}
\end{beschreibung}
\begin{beispiel}
 \setfnpct{mult-fn-delim=//}
 \begin{minipage}{.4\linewidth}
  \noindent The three little pigs built their houses
  out of straw\footnote{not to be confused with hay},
  sticks\footnote{or lumber according to some sources}
  and bricks\multfootnote{probably fired clay bricks;
  or something else//what do I know}.
 \end{minipage}
\end{beispiel}

\subsection{Automagic}
If you want you can turn all footnotes into \cmd{multfootnote}s.
\begin{beschreibung}
 \option{multiple}{\default{true}/false} let \cmd{footnote} behave like \cmd{multfootnote}.
 \emph{This option can only be set in the preamble}. Default = \code{false}
\end{beschreibung}
\begin{beispiel}
 % in preamble: \setfnpct{multiple}
 \begin{minipage}{.4\linewidth}
  \noindent The three little pigs built their houses
  out of straw\footnote{not to be confused with hay},
  sticks\footnote{or lumber according to some sources}
  and bricks\footnote{probably fired clay bricks;or
  something else}.
 \end{minipage}
\end{beispiel}

\section{Footcites}
Preliminary testing suggests that \fnpct seems to be compatible with the
\cmd{footcite} and \cmd{footfullcite} commands provided by the \paket{biblatex}
package. However, the punctuation switching does not apply to them.

The kerning works \ldots
\begin{beispiel}
 \setfnpct{before-footnote-space=2pt}
 \begin{minipage}{.5\linewidth}
  \noindent The three little pigs built their houses
  out of straw\footnote{not to be confused with hay},
  sticks\footnote{or lumber according to some sources}
  and bricks\footnote{probably fired clay bricks}. The
  companion\footfullcite{companion} has nothing on this
  topic.
 \end{minipage}
\end{beispiel}

\ldots\ but the punctuation switching doesn't:
\begin{beispiel}
 \begin{minipage}{.5\linewidth}
  \noindent The three little pigs built their houses
  out of straw\footnote{not to be confused with hay},
  sticks\footnote{or lumber according to some sources}
  and bricks\footnote{probably fired clay bricks}. The
  companion has nothing on this topic\footfullcite{companion}.
 \end{minipage}
\end{beispiel}

One can simulate that behaviour by actively setting one of these commands:
\begin{beschreibung}
 \befehl{kfp} insert the \key{after-dot-space} or (if \key{punct-after}{true})
 the \key{before-dot-space}.
 \befehl{kfc} insert the \key{after-comma-space} or (if \key{punct-after}{true})
 the \key{before-comma-space}.
\end{beschreibung}
\begin{beispiel}
 \setfnpct{after-punct-space=-.15em}
 \begin{minipage}{.5\linewidth}
  \noindent The three little pigs built their houses
  out of straw\footnote{not to be confused with hay},
  sticks\footnote{or lumber according to some sources}
  and bricks\footnote{probably fired clay bricks}. The
  companion has nothing on this topic.\kfp\footfullcite{companion}
 \end{minipage}
\end{beispiel}

In case of \key{punct-after}{true}:
\begin{beispiel}
 \setfnpct{punct-after}
 \begin{minipage}{.5\linewidth}
  \noindent The three little pigs built their houses
  out of straw\footnote{not to be confused with hay},
  sticks\footnote{or lumber according to some sources}
  and bricks\footnote{probably fired clay bricks}. The
  companion has nothing on this topic\footfullcite{companion}\kfp.
 \end{minipage}
\end{beispiel}

\appendix
\printbibliography
\printindex
\end{document}